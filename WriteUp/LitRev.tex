\chapter{Literature Review}



The economics of flexibility solutions in power systems, especially electric energy storage (EES), is an active topic in research. It has drawn great attentions from the academics, investors and policy makers. 


% electricity storage are currently in the focus of research, by academics, utilities, potential investors as well as policy makers. The present document is the result of the analysis of more than 200 publications on that subject. It aims at presenting the “state of the art” regarding research on the economics of electricity storage. Three particular aspects are given attention to: the methodologies used, the profitability results obtained and the impact of regulation on storage economics.
\section{Purpose and stakeholder}

\section{Modelling methodology}
\subsection{Overview}
Engineering vs system
Linear vs nonlinear
Deterministic vs stochastic problems
Solving techniques

\subsection{Engineering model}
Price taker
perfect forecast
stochastic or dynamic programming
Hybrid system
Service mutualization

\subsection{System model}


\section{Affecting factor}
\subsection{Techno-economic characteristics of power system}
\subsubsection{Generation}
Generation mix (Renewable integration)
Fuel Prices
\subsubsection{Climate and weather}

\subsubsection{Transmission}
Grid topology
Transmission capacity

\subsubsection{Consumption}


\subsection{Power market degisn and policy regulation}
\subsubsection{Player and competitive landscape}

\subsubsection{Renewable Support Scheme}

\subsubsection{Power Market Design}
Market structure and rules: nodal, interval, reserve market
Access

In general, the seven ISOs/RTOs require companies that service loads (i.e., the energy re- quirements of end-use customers) to provide reserves in proportion to their loads. (ref to Project Report: A Survey of Operating Reserve Markets in U.S. ISO/RTO-managed Electric Energy Regions)


\subsubsection{Ownership and dispatch}

\subsubsection{Direct policy support}
Capacity market
Feed-in premium or tariff
Other program

\section{Value of results for reference}
\subsection{Demand for flexiblity in power system}

\subsection{Profitability of flexibility solutions}
