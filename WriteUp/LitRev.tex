\chapter{Literature Review}
\label{ch:LitRev}


As is clearly revealed by the literuare review, there exist abused research articles generally on this topics of flexibility management. However, there exist very few academic works that serves the needs of our target audiences who are the management of technology vendors. The deviationsof interests result in gaps that make it difficult to directly use the existing works. These gaps include:

\begin{itemize}
	\item Most of the researches are based on one specific technology and one specific market, as usually a utility company or a grid planner is operating in one market regimes and a technical professional is focusing on one technology.  However, our target audiences are likely to be interested in various markets and technologies.
	\item Scope
	\item Mothed - proof of concept
\end{itemize}




Conventionally, their decision makings are supported primarily by commercial consulting firms who relied much on qualitative anlysis or quantitative data-anlytics. Even when sometimes it is possible that those firms have developed model with fundamental and physical approach, the model is always customized and not public 

most of the researches are focusing one specific technology and one specific market, due to the nature of their target audiences. However, the managment iof a technology vendor will likely to be interested in various markets and various technologies. 

The economics of flexibility solutions in power systems, especially electric energy storage (EES), is an active topic in research. It has drawn great attentions from the academics, investors and policy makers. 


% electricity storage are currently in the focus of research, by academics, utilities, potential investors as well as policy makers. The present document is the result of the analysis of more than 200 publications on that subject. It aims at presenting the “state of the art” regarding research on the economics of electricity storage. Three particular aspects are given attention to: the methodologies used, the profitability results obtained and the impact of regulation on storage economics.
\section{Purpose and stakeholder}


\section{Modelling methodology}
%Two broad approaches have been taken to modelling UK electricity systems. As discussed in Section 2.2.1, system studies tend to employ a holistic and system wide perspective with only coarse temporal resolution. Other studies attempt to understand system balancing with high penetration of wind and issues arising from ramp and slew rates of wind and errors in wind forecasting. These studies require high temporal resol- ution and therefore tend to simulate short periods of time and make static assumptions for system context.(Black and Strbac, 2007, 2006; Pelacchi and Poli, 2010; Barton and Infield, 2004; Bathurst and Strbac, 2003)
\subsection{Overview}
Engineering vs system
Linear vs nonlinear
Deterministic vs stochastic problems
Solving techniques

\subsection{Engineering model}
Price taker
perfect forecast
stochastic or dynamic programming
Hybrid system
Service mutualization

\subsection{System model}


\section{Affecting factor}
\subsection{Techno-economic characteristics of power system}
\subsubsection{Generation}
Generation mix (Renewable integration)
Fuel Prices
\subsubsection{Climate and weather}

\subsubsection{Transmission}
Grid topology
Transmission capacity

\subsubsection{Consumption}

\subsubsection{Merit-order model}
\cite{Sensfuss2008}

\cite{SaenzdeMiera2008}
\cite{Tveten2013}
\cite{McConnell2013}
\cite{Gelabert2011}
\cite{Clo2015}
\cite{Woo2016}
\cite{Cludius2014}
\cite{He2013}
\cite{Mulder2013}

\subsection{Statistic model}
\cite{Alipour2017}
%Efficient modelling of 

\subsection{Perfect forecast}
\cite{He2011}
\cite{Sioshansi2009}
\cite{Bathurst2003}
\cite{Drury2011}
\cite{Connolly2012}

\subsection{Power market degisn and policy regulation}
\subsubsection{Player and competitive landscape}

\subsubsection{Renewable Support Scheme}

\subsubsection{Power Market Design}
Market structure and rules: nodal, interval, reserve market
Access

In general, the seven ISOs/RTOs require companies that service loads (i.e., the energy re- quirements of end-use customers) to provide reserves in proportion to their loads. (ref to Project Report: A Survey of Operating Reserve Markets in U.S. ISO/RTO-managed Electric Energy Regions)

Balancing market design \cite{Wartsila2014} \cite{Moller2010}

\subsubsection{Ownership and dispatch}

\subsubsection{Direct policy support}
Capacity market
Feed-in premium or tariff
Other program

\section{Value of results for reference}
\subsection{Demand for flexiblity in power system}

\subsection{Profitability of flexibility solutions}
