\chapter{Power Markets and The Role of Flexibility Solutions: An Analytical Framework}
\chaptermark{Market regimes}
\label{ch:market}
\textit{This chapter aims at offering a comparative view on different power market regimes, based on which an analytical framework can be established. Such a framework offers technology vendors a solid foundation for qualitatively analyzing the opportunities of flexibility solutions in a given market context. By mapping a list of mature power markets worldwide, we extract some key attributes of power market structures that impact the value of flexibility solutions.}

\section{Motivation for a power market analysis framework}

Conceived in the 1980s and facilitated in the 1990s, liberalization of power markets has become the mainstream worldwide \cite{Srivastava2011,Ranci2013,Vagliasindi2013}. However, different conditions exist across economies including historical, political and climatic factors. As a result, structures of these power markets tend to be very heterogeneous. %Moreover, with development of technologies, power markets face pending or undergoing restructuring, make them a rapidly changing field of the economy. \cite{Ziel2015}
This brings great challenges to companies that pursue a cross-regional or even global footprint, since business models for flexibility solutions as well as their feasibility and performance depend extensively on the power market structure. Compared to other stakeholders that are interested in flexibility solutions such as utilities and regulators, technology vendors are more likely to have international ambitions. This is not only because they have fewer regulatory barriers, but also because firms with higher research and development (R\&D) intensity have stronger motivation for expanding geographic boundaries to mitigate market risks and seek growth opportunities \cite{Brouthers2007}.

Therefore, in this chapter we map the taxonomy of power markets with a particular focus on characteristics related to flexibility solutions, in order to provide a general framework for technology vendors facing different power markets.

Such a global view is established by generalizing and comparing market regimes in different systems that are listed in Table \ref{tab:markets}. We will name a few of them as typical examples while discussing each structural attribute. However, it shall be noted that the goal of this chapter is not to provide comprehensive analysis on each of the system. With on-going restructuring of market regimes, each system is constantly evolving over time. Taking the electricity market in Great Britain (GB) as an example, it had been operating in the model of power pool for over 10 years before it reformed to a power exchange arrangement in 2001 \cite{Rebours2009,FrontierEconomics2016,ofgem_m}, and in a more recent restructuring in 2014 they established the capacity market that did not exist there before \cite{ofgem_cm}. 

Nevertheless, our general framework will remain largely stable regardless of adjustments in individual markets. This again reveals the importance of an analytical framework facing such a fast-changing area. Using the same example in GB, readers can immediately identify potential opportunities riding on the introduction of capacity market by referring to Section \ref{sec:CM}.


\begin{table}[h!]
	\small
	\centering
	\begin{tabular}{L{5cm} l l l}
		\hline
		\hline
		\textbf{System} & \textbf{Abbreviation} & \textbf{Country} & \textbf{Main Reference} \\
		\hline
		\hline
		PJM Interconnection & PJM & US & \cite{Rebours2009,Srivastava2011,Cochran2013,EllisonJ.F.TesfatsionL.S.LooseV.W.Byrne2012,Gilstrap2015,Brown2015,Borenstein2015,PJM_web,PJM2017b,PJM2017c}\\
		\hline
		New York ISO & NYISO & US & \cite{Cochran2013,EllisonJ.F.TesfatsionL.S.LooseV.W.Byrne2012,Gilstrap2015,Borenstein2015,NYISO_web}\\
		\hline
		Midcontinent ISO\footnote{Formerly named Midwest ISO}& MISO & US \& Canada& \cite{EllisonJ.F.TesfatsionL.S.LooseV.W.Byrne2012,Gilstrap2015,Borenstein2015,MISO_web}\\
		\hline
		ISO New England & ISO-NE & US & \cite{EllisonJ.F.TesfatsionL.S.LooseV.W.Byrne2012,Gilstrap2015,Borenstein2015,ISO_NE_web}\\
		\hline
		California ISO & CAISO & US& \cite{Rebours2009,EllisonJ.F.TesfatsionL.S.LooseV.W.Byrne2012,Gilstrap2015,Borenstein2015,CAISO_web}\\
		\hline
		Southwest Power Pool & SPP & US & \cite{EllisonJ.F.TesfatsionL.S.LooseV.W.Byrne2012,Gilstrap2015,Borenstein2015,SPP_web}\\
		\hline
		Electric Reliability Council of Texas & ERCOT & US & \cite{Srivastava2011,EllisonJ.F.TesfatsionL.S.LooseV.W.Byrne2012,Brown2015,Gilstrap2015,Borenstein2015,ERCOT_web}\\
		\hline
		Ontario Independent Electricity System Operator& IESO/ Ontario& Canada & \cite{Cochran2013,Brown2015,Ontario_web}\\
		\hline
		Alberta Electric System Operator & AESO/Alberta & Canada &  \cite{Brown2015,Alberta_web}\\
		\hline
		National Electricity Market (Australia) & NEM & Australia & \cite{Srivastava2011,Brown2015,AEMO2010,AEMO2015a}\\
		\hline
		National Electricity Market of Singapore & NEMS & Singapore & \cite{Brown2015} \\
		\hline
		Germany\footnote{Referring to territories of 4 TSOs, Tennet, Amprion, 50 Hertz, TransnetBW under the regulation of Bundesnetzagentur (BNetzA) with large volume of electricity traded OTC and on power exchange, EPEX SPOT. } & DE & Germany & \cite{FrontierEconomics2016,Wartsila2014,ConsentecGmbH2014,Deloitte2015} \\
		\hline
		Single Energy Market (Ireland) & SEM/ Ireland & Ireland & \cite{FrontierEconomics2016,Cochran2013}\\
		\hline
		Great Britain\footnote{Referring the territory of the TSO, National Grid, under the regulation of Office of Gas and Electricity Markets (ofgem) with large volume of electricity traded OTC and on power exchange, APX Power UK and N2EX .} & GB & Great Britain & \cite{Rebours2009,FrontierEconomics2016,ofgem_cm,ofgem_m,EnergyUK2017} \\
		\hline
		Other European Markets & - & - & \cite{FrontierEconomics2016}\\
		\hline
		\hline
	\end{tabular}
	\caption{List of markets involved in this study} \label{tab:markets}
\end{table}

In Chapter \ref{ch:introduction}, we identified three applications of flexibility solutions in wholesale markets, including:

\begin{itemize}
	\item \textbf{Arbitrage in energy market}, and
	\item \textbf{Frequency control in ancillary services market}, and
	\item \textbf{Supply adequacy in capacity market}. 
\end{itemize}

Correspondingly, we systemically investigate how the feasibility of these applications is influenced by different market regimes, i.e. structure of energy/ ancillary service/ capacity markets, in the reminder of this chapter. Unlike many other studies that are also focused on comparison of different market structures but for the reference of market designers, we do not analyze the full rationale behind the market design nor their comprehensive merits and drawbacks. Instead, this chapter is focused only on the differences themselves and their direct impacts on value of flexibility solutions.


~\newpage

\section[Flexibility solutions in energy markets]{Flexibility solutions in energy markets%}
	\sectionmark{Energy market}}
\sectionmark{Energy market}
\label{sec:market-energy}
We start our analysis from the wholesale energy market as it constitutes the central transaction platform in power markets \cite{Cochran2013}. 

In a competitive market price should act as an effective signal to coordinate the balance of supply and demand. Reflecting this principle in energy markets, if a market is well-designed, price volatility would increase due to lack of flexibility and in turn become an incentive to encourage participation of new flexibility sources, as introduced in Chapter \ref{ch:introduction}. However, it is not always the case in reality since power market design takes into consideration for not only economic but also physical and political factors. Moreover, since market design is likely to lag behind technological development, some legacy rules tend to create barriers for new technologies even if they may already be favored by those physical, economic and political requirements.

Therefore, although energy arbitrage that can absorb energy in supply surplus and release energy in supply shortage is theoretically beneficial to power systems, it is not always feasible depending on market rules. 

\subsection{Market model: Power pool vs. power exchange}

First of all, it is worthwhile to point out the difference between power pool and power exchange, since they represent two fundamentally distinct approaches of how power markets are organized. 

\begin{figure}[h!]
	\centering
	\includegraphics[width=0.95\linewidth]{Figures/PowerPoolExchange}
	\caption{Illustration of difference between power pool and power exchange}
	\label{fig:pppx}
\end{figure}

As shown in Figure \ref{fig:pppx}, in the model of power pool, all the structural components of power markets are integrated and coordinated by a single entity that is both market operator and system operator  \cite{Srivastava2011,Barroso2005}, often named independent system operator (ISO). Since scheduling is an integral part of the power market, schedules are determined through a single market gateway, and markets are cleared abiding by the limits of physical deliveries.  In a power pool, ISO seeks to minimize the system total production cost through a centralized unit commitment to fulfill demands economically. Generators must follow the commitment schedule and the dispatch instructions issued by the ISO to receive payments \cite{Kardakos2013}. Otherwise, ISO may charge penalties from the generators or suspend their participation in the power pool. Market activities are mainly on the generation-side, while demands are consolidated as input of ISOs' optimization. Players on the demand-side are usually not able to participate in the market directly unless specific measures are implemented.

In contrast, in the model of power exchange, a transmission system operator (TSO) is still responsible for scheduling coordination, ancillary service provision and transmission system operation, but power transactions are made through a power exchange organized by a third party or through bilateral contracts. Therefore, a market participant is able make electricity transactions in more than one market. As a matter of fact, power exchanges are mostly established by profit-seeking market players and have evolved from the bilateral contract model \cite{Barroso2005}. The system operator usually has no direct control on the power exchange and its role is limited to the physical aspect of maintaining system security. Each producer is responsible for self-scheduling its own units with a decentralized price-based unit commitment \cite{Kardakos2013}. Therefore, power markets organized in power exchange model can be viewed to have a higher level of unbundling than those in power pool model, without invention of physical system operators in electricity trading activities.

Examples of energy markets organized in power pool:
	\begin{itemize}
		\item Most markets organized by ISOs in North America such as PJM, NYISO, Alberta, etc.
		\item Australia's NEM
		\item Ireland's SEM.
	\end{itemize}

Examples of energy markets organized in power exchange
	\begin{itemize}
		\item most markets in European countries such as Germany, Nord Pool, GB etc.
		\item CAISO in the US.
	\end{itemize}

\subsubsection{Implications for flexibility solutions}
In power exchange, the participation from supply-side and demand-side is generally symmetric and offers/bids are usually in the single form of price-quantity pairs. The physical realization of delivery is unbundled  from market activities and is not concerned by market operators. This allows great freedom for flexibility players to participate in the market, regardless of whether the flexibility comes from supply- or demand- side or mixed, and which technologies are employed. 

In power pool, however, generators are usually required to submit complex unit offers including physical information of resources, e.g., unit start-up and shut-down procedures, minimum-up/down time constraints, min/max power output restrictions, ramp-rate limits, transmission limits etc. \cite{Kardakos2013}, and participation from demand-side is generally limited. Therefore, with bundling physical and market activities, participation of flexibility is extensively under control of power pool operators. Being recognized as a generation resource or special market gateway for demand-side participation is necessary prerequisite for a new flexibility resource to directly participate in markets. Otherwise, it would be only limited to behind-the-meter applications where some flexibility resources such energy storage can complement with existing resource or load to adjust a player's position in market. In this way, the operation of flexibility might not be optimal and aggregation is impossible. In addition, due to the strong position of power pool operators, it is less likely for players to gain market power than in power exchange.

Overall, there are greater limits for market participation of flexibility solutions in power pool than in power exchange. Technology vendors need to go further in their efforts aligning market rules and regulatory environment for business planning in power pools. 

\subsection{Marketplace}

In most regions, the energy market consists of several marketplaces along the timeline, as shown by Figure \ref{fig:energy-marketplaces}\footnote{Forward products are excluded since financial derivative markets are out of our scope; refer to Chapter \ref{ch:introduction}.}.

\begin{figure}[h!]
	\centering
	\includegraphics[width=0.95\linewidth]{Figures/EnergyMarketplaces}
	\caption{Typical marketplaces in wholesale energy market}
	\label{fig:energy-marketplaces}
\end{figure}

In markets organized with power exchange model, large volumes of energy are usually traded in day-ahead (DA) market. Intra-day (ID) market, which can be viewed as an extension of day-ahead spot market bringing gate closure near delivery, is a common measure to mitigate increasing needs of real-time balancing operations, as introduced in Chapter \ref{ch:introduction}. All deviations from the commitment scheduled by DA and ID markets requires balancing energy delivery that is coordinated by system operators and are settled through a third marketplace, often named balancing energy market. Imbalance settlements that are accounted in the balancing energy market involve two mechanisms: first, the deviation of one player can be somehow offset by opposite deviations of other players; second, on the system level, the aggregated imbalance is settled by activating frequency control services. In most market regimes, system operators will play a centralized role to clear and settle costs incurred from both mechanisms, while sometimes system operators may allow ex-post trading between market players regarding the imbalance settlement through the first mechanism such as the Swiss and Greek power markets.

Slightly different arrangements are adopted in power pools. Since delivering balancing energy is the responsibility of the  same entity that operates the energy markets, real-time markets are used for settlements of both post-DA scheduling adjustments and balancing operations.

The three-settlement market (i.e. day-ahead, intra-day and balancing market) is the European Union target electricity model \cite{EuropeanCommission2016} so has been implemented in most European energy markets such as Germany, France, Denmark, GB, Italy, Spain, etc.  Two-settlement market (i.e. day-ahead and real-time market), on the other hand, is a common practice in North America  \cite{Cochran2013}.

%\subsubsection{Implications for flexibility management}
Generally, arbitrage in DA market is less favorable for emerging flexibility players, due to relative low volatility and dominance of large conventional generation companies. Flexibility solutions shall gain more advantage in market closer to delivery due to its comparative competence of fast response and operations to conventional generators. 

Nonetheless, participation in any marketplace to perform arbitrage is potentially profit-making. Therefore, identifying which marketplaces exist and whether they are accessible is a necessary step for valuing the opportunities of flexibility solutions.

\subsection{Pricing scheme}
If a marketplace is accessible for flexibility players, a further concern would be the profitability of arbitrage. Since arbitrage is essentially a game played with prices, the pricing mechanism is of most importance, which is however highly diverse across different markets.

\subsubsection{Nodal pricing vs. zonal pricing}
With nodal pricing scheme, prices at each network node are different. On the contrary, uniform pricing scheme applies same price everywhere in the whole control area. Zonal pricing as a trade-off between these two schemes, use the same price in a particular zone including a bundle of nodes.

Nodal pricing internalizes network congestion in price formation. If congestion restricts lowest-cost electricity being transmitted to a particular location, electricity with higher cost but no congestion is dispatched and consequently price at that location will rise. Nodal pricing has clear benefits \cite{Wang2015} but it is harder to implement, especially in markets arranged in power exchange where market operators have no insight into the physical system\footnote{In these regions, it is possible to implement nodal pricing in balancing markets that is coordinated by physical system operators, as illustrated by a research project \cite{Ecogrid}. However, we have not seen any large-scale practice in reality.}. 

Nodal pricing is adopted in many systems in North America, such as PJM, CAISO, NYISO, ISO-NE etc., using a mechanism named locational marginal price (LMP) model. Zonal pricing is used in Australia's NEM and other energy markets organized in power exchange model. 

Nodal pricing incorporates the consideration of congestion. The value of T\&D congestion relief can theoretically be partially captured by arbitrage, especially using flexibility technologies that are easier to be deployed at smaller scale in particular locations such as batteries. However, for aggregators, nodal pricing increases the operational complexity.

\subsubsection{Time resolution}
Since RES generation is intermittent and may vary significantly in a short time interval so may the residual load. As a result, a higher time resolution of pricing can better represent the market need for flexibility. Emerging flexibility solutions with faster response and higher ramp rate shall gain advantages with higher pricing resolution in theory. However, it should be noted that the pricing and dispatching time interval is sometimes different to the settlement interval. For example, the real-time markets in PJM has 5-min pricing resolution but settlement of energy delivery is accounted at hourly resolution \cite{PJM2017}. In such an arrangement, arbitrage against the original price signals may be activated for price differences within a settlement interval, which brings no revenue. Therefore, the operational plan of arbitrage should be determined based on estimation of prices for actual settlement.


\section[Flexibility solutions in ancillary service markets]{Flexibility solutions in ancillary service markets%}
	\sectionmark{Ancillary service market}}
\sectionmark{Ancillary service market}
\label{sec:market-as}
Among all ancillary services, this thesis is particularly focused on frequency control services that are used to tackle imbalance between supply and demand by delivering balancing energy, as introduced in Chapter \ref{ch:introduction}. Frequency control services are usually the most costly among all ancillary services and relying on services provision from market players, while there are usually no markets for other ancillary services such as voltage support, loss compensation, black start etc \cite{Rebours2009,Cochran2013}. 

In different regimes, there exist many differences regarding how frequency control services are defined, procured and operated, as well how the cost is allocated and recovered. Understanding these differences allows technology vendors know which services can be provided using flexibility and to whom they can sell flexibility solutions.

\subsection{Terminology for frequency control services}
Different terminologies used in different power jurisdictions may easily lead to confusion while comparison between different regimes is to be made. Different terms are often used to refer to the same service, while in some instances the similar terms may refer to two disparate services in different regimes. For example, secondary control reserve (SCR) and automatic frequency restoration reserve (aFRR) (both used in European markets) are interchangeable concepts. On the contrary, primary reserve in North America is often used to distinguish services from supplementary reserve, while it is closer to the concept of teirtiary reserve rather than primary reserve used in Europe.

Generally, these terminologies can be classified into two groups as they follow the guidance of service definitions from the Federal Energy Regulatory Commission (FERC) and the Union for the Coordination of the Transmission of Electricity (UCTE). According to the functioning mechanism\footnote{ PCR refers to response activated locally by a speed governor fitted in generator. SCR is activated by a centralized control signal named automatic generation control (AGC) signal. TCR follows manual orders from system operators \cite{EllisonJ.F.TesfatsionL.S.LooseV.W.Byrne2012}. }, terminologies in these two systems can be mapped into a a comparison framework show by Table \ref{tab:term}.

\begin{table}[h!]
	\footnotesize
	\centering
	\begin{tabular}{L{3cm} L{3cm} | L{3cm} L{3cm}}
		\hline
		\hline
		\textbf{UCTE terms} &  \textit{Equivalents} & \textbf{FERC terms} &  \textit{Equivalents} \\
		\hline
		\hline
		Primary control reserve (PCR) & Frequency containment reserve (FCR) & Frequency response & \\
		\hline
		Secondary control reserve (SCR) & Automatic frequency restoration reserve (aFRR) & Frequency regulation & \\
		\hline
		\multirow{3}{3cm}{Tertiary control reserve (TCR)}
		  & \multirow{3}{3cm}{Manual frequency restoration reserve (mFRR)} & Spinning reserve & Synchronous reserve \\
		  \multirow{3}{3cm}{}& \multirow{3}{3cm}{}& Non-spinning reserve & Non-synchronous reserve/ Quick-start reserve \\
		  \multirow{3}{3cm}{}&\multirow{3}{3cm}{} & Supplemental reserve & Replacement reserve \\
		  \hline
		  \hline
	\end{tabular}
\caption{Terminology for frequency control reserves in various regimes \cite{Rebours2009,EllisonJ.F.TesfatsionL.S.LooseV.W.Byrne2012,Wang2015}}\label{tab:term}
\end{table}

It shall be noted that in terms of activation time, UCTE has specifically defined that:

\begin{itemize}
	\item Primary reserve shall be automatically activated within 30s;
	\item Secondary reserve need to be completely delivered within 15 minutes; 
	\item Tertiary reserve shall start within 15-20 minutes after received the order from system operators.
\end{itemize}

In contrast, the time framework for each service category is not aligned among markets in North America \cite{EllisonJ.F.TesfatsionL.S.LooseV.W.Byrne2012}, but generally, activation time of frequency regulation is comparable to an in-between state of primary and secondary control reserve. 

In addition, there are no markets for frequency response in North America \cite{Rebours2009,EllisonJ.F.TesfatsionL.S.LooseV.W.Byrne2012} that are equivalent to primary control reserve markets in Europe.

Generally, new flexibility solutions have advantages for services with shorter activation time and shorter duration compared to service providers using conventional generation. Therefore, frequency control services can be roughly ranked in accordance with the extent to which they are suited to emerging technologies, from most to least: primary, secondary and tertiary. However, this is case-specific depending on characteristics of specific technologies and markets, so is not discussed in details here.

\subsection{Procurement and cost allocation}

Usually, markets for frequency control services involve trading for two commodities, i.e. capacity and energy, as shown by Figure \ref{fig:FCR_market}. 

\begin{figure}[h!]
	\centering
	\includegraphics[width=0.95\linewidth]{Figures/FCR_market}
	\caption{Illustration of markets and activities related to frequency control services}
	\label{fig:FCR_market}
\end{figure}

Capacity refers to a commitment that service providers make to system operators, that they will keep reserves ready to be dispatched for real-time operations. The requirement for capacity is determined by the system operator and procured ahead of real-time operation. Specifically, in the continental European synchronously interconnected system, a total PCR of 3000 MW needs to be provided according to the rules of the European Network of Transmission System Operator (ENTSO-E), while amounts of necessary SCR and TCR capacity are determined by each TSO \cite{ENTSO-e_handbook}. For instance, in Germany, TSOs run a quarterly assessment process to dimension the provision of SCR and TCR for next three months \cite{ConsentecGmbH2014}. In North America, ISOs determine the need for reserve capacity by conducting their own processes, so-called reliability assessment, which take place after gate closure of day-ahead market and before each operating hour\cite{EllisonJ.F.TesfatsionL.S.LooseV.W.Byrne2012}. 

Energy is what services providers actually deliver to the system upon activation by system operators in real time. The amount of energy is determined based on physical needs for grid balancing.

The acquisition and settlement process for the frequency control capacity and energy also varies amongst different market regimes.

Generally, two models are identified. We name them as centralized procurement and decentralized procurement respectively; see Figure \ref{fig:FCR_market-model}.

\begin{figure}[h!]
	\centering
	\includegraphics[width=1.05\linewidth]{Figures/FCR_market_model}
	\caption{Two models for procurement and cost allocation of frequency control services}
	\label{fig:FCR_market-model}
\end{figure}

In the centralized model, the system operator is designated as the single buyer \cite{Rebours2009}. System operators (SO) will either organize auctions in short term ahead of the operating day, e.g. German TSOs organize weekly-ahead auctions for SCR and day-ahead auctions for TCR \cite{ConsentecGmbH2014}, or seek long-term bilateral contract with service providers, e.g. Australian Energy Market Operator (AEMO) uses this approach to organize ancillary services in NEM \cite{AEMO2015}. On the other hand, SOs need to recover costs incurred by charging entities with obligation. In different markets and for disparate services, obligations are assigned in various ways. For example, costs for energy of frequency control services in Germany and for regulation reserve in Australia are recovered from entities who violate their commitments determined in energy markets, while costs for capacity of frequency control services in Germany and for contingency reserve (similar to PCR) in Australia are socialized among all market participants. 

Decentralized model is adopted by ISOs in North America. In this model, ISOs allocate requirements for reserve capacity to market players according to their servicing loads to the system total load \cite{Rebours2009,EllisonJ.F.TesfatsionL.S.LooseV.W.Byrne2012,PJM2017b}. Market players have to fulfill their own obligations through self-supplied reserve, through bilateral contract with other market participants, and/or through purchases of reserve in some form of reserve market organized by ISO \cite{EllisonJ.F.TesfatsionL.S.LooseV.W.Byrne2012}. In this way, market participants in the power market are put into competition for procuring frequency control services. Examples using this model include all seven ISOs in the US.

Flexibility solutions can be employed for the provision of frequency control services and for fulfilling obligations in both arrangements. However, while provision and obligation fulfillment are symmetric in the decentralized approach, there might be asymmetry in the centralized approach with SOs standing in-between. Payments may differ between services provided for SOs and for market players to fulfill their obligations.

\subsection{Frequency control product design}
Further to the high-level distinctions mentioned previously, attentions should also be paid to some key details regarding how the frequency control service as products are designed. Product design will significantly affect the feasibility and profitability of certain technologies providing frequency control services. Without mentioning too many technological specifications, we discuss four points here. 

First of all, pre-qualification of resources to provide a given service is necessary. While activation time is usually an advantage of emerging flexibility solutions, duration of dispatch tends to be a bottleneck, especially for tertiary control reserves. For instance, CAISO requires a minimum of 30 minutes duration for delivering spinning and non-spinning reserves and duration of providing tertiary reserve for German TSOs is in 6-hour blocks. In these cases, some flexibility solutions, such as flywheel energy storage that is only able to last for about 15 minutes \cite{EllisonJ.F.TesfatsionL.S.LooseV.W.Byrne2012,Beaudin2014}, are excluded from provision of those services.

Second, frequency control services are sometimes divided into up and down services. Up services mean there are generation shortage and injection of energy or reduction of demand are required. On the contrary, down services refer to situations where more demand or less generation is needed. Separate markets for these two types of services would allow more choices for flexibility players to make optimal offers in accordance of the technological characteristics of their flexibility resources.

Besides, it is of a concern how automatic frequency control signals are engineered. For example, an energy storage device that does not generate energy will favor a signal that is energy-neutral to it, i.e. the state of charge of the device can come back to its initial value after a period of operation.

Finally, one should consider how services are priced. Capacity commitment and actual delivery, i.e. amount of released energy and sometimes performance as well, are normally priced and settled separately. Since flexibility solutions have the potential to outperform conventional flexibility solutions considering their technological characteristics of fast response and high ramp rate, a pricing scheme where performance of delivery is valued tends to offer merits for emerging flexibility solutions. By this rationale, the FERC requires ISO markets to compensate for regulation based on actual service provided
according to its Order 755 \cite{FERC755}. Some ISOs including PJM, NYISO, ISO-NE followed the order to establish such a mechanism. Nevertheless, in most market regimes, only amount of energy is accounted for final payment for frequency control services.

More detailed impact of product design and  technical implications will be discussed in Chapter \ref{ch:cases}.

\section[Flexibility solutions in capacity markets]{Flexibility solutions in capacity markets%}
	\sectionmark{Capacity Market}}
\sectionmark{Capacity Market}
\label{sec:CM}

The capacity market is established in some power market jurisdictions to minimize investment risks of power generators so that resource adequacy can be effectively ensured. Investors are remunerated for commitment to keep capacity online. However, it is not a common practice, because of complex political reasons which are not our focus in this thesis, but it is worth to mention briefly that ensuring minimal investment risk for generators means risks are somehow shifted to consumers \cite{Cochran2013}. 

However, for flexibility players and technology vendors, the existence of capacity market is generally favorable as it potentially provides a direct revenue stream. Naturally, one should examine which technologies are suitable and whether demand-side resources are qualified to receive remuneration.

Examples of power market jurisdictions with capacity market include PJM, NYISO, ISO-NE, Spain, Ireland, GB (since 2014), etc. Also, transition from an energy-only market towards a capacity market has been observed in some markets, e.g. Ontario IESO and Alberta AESO are in the process of developing a capacity market, initiated in 2014 and 2016 respectively. 

In energy-only markets, system operators have sometimes taken alternative measures to ensure adequacy, e.g. strategic reserves or named emergency products. Strategic reserves and emergency products are either generation capacity or curtailable loads that are activated only when scarcity of generation is observed (typically reflected by extremely high prices).

Energy-only markets with such capacity remuneration mechanisms include: ERCOT, Australia's NEM, Germany, Nord Pool, Belgium etc. 

Finally, for markets without any of these capacity measures mentioned above, it is likely for extreme prices to occur, which will be an indirect incentive for flexibility players' arbitrage in energy markets.

\section{Aggregator and demand-side participation}

Participation of aggregators and other providers of demand-side flexibility is not always allowed in some marketplaces. This is especially an issue for ancillary services and capacity markets as they were initially designated only for generation resources. Participation in energy markets may also be limited in power pool arrangement as discussed previously. Therefore, it is of great importance to examine the market rules regarding this issue.

So far, examples of power pools that allow participation of aggregators and demand-side responses in energy market include: PJM, ISO-NE, Ontario IESO, Singapore and Australia's NEM (as of April 2017 \cite{AEMO_DR}), etc.

Examples of jurisdictions allowing participation of aggregators and demand-side responses in frequency control ancillary service market include: PJM, Ontario, Singapore, Alberta AESO, ERCOT, Australia's NEM, etc.

Examples of capacity markets that have remuneration programs for demand-side resources include: PJM, ISO-NE.

Examples of energy-only market that have strategic reserve of emergency products for demand-side resources include: ERCOT, Australia's NEM, Nord Pool, Germany, etc.

\section{Summary and the analytical framework}

From the analysis presented above, it is clearly seen that investigating opportunities for flexibility solutions across different market regimes is indeed a sophisticated task since many layers of hierarchy exist in terms of structural differences across markets. In order to better guide technology vendors for qualitative assessment of flexibility solution in a given regime, we organized the previous analysis into analytical frameworks illustrated by Figure \ref{fig:qualitative-energy}-\ref{fig:qualitative-capacity}. We apply these frameworks for our own analysis in the reminder of this thesis.

\begin{figure}[h!]
	\centering
	\includegraphics[width=0.95\linewidth]{Figures/Q_energy}
	\caption{Analytical framework for qualitative analysis of flexibility solutions in energy market}
	\label{fig:qualitative-energy}
\end{figure}

\begin{figure}[h!]
	\centering
	\includegraphics[width=0.95\linewidth]{Figures/Q_frequency_control}
	\caption{Analytical framework for qualitative analysis of flexibility solutions in frequency control market}
	\label{fig:qualitative-fr}
\end{figure}

\begin{figure}[h!]
	\centering
	\includegraphics[width=0.95\linewidth]{Figures/Q_capacity}
	\caption{Analytical framework for qualitative analysis of flexibility solutions in capacity market}
	\label{fig:qualitative-capacity}
\end{figure}

