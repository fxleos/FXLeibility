\chapter{Power Market Framework and Proposed Business Model for Flexibility Management in Selected Segments and Geographies}
\label{ch:market}
\section{Power market framework}

\subsection{PJM}

\subsection{Germany}
\label{market:germany}

\subsection{Balancing Energy Market}

%Market participants in Germany are organised into balancing groups, known as Bilanzkreise (BK). BK can range from individual large generators to aggregations of smaller renewable generations, to a Stadwerke representing large portions of aggregated demand.

%Every balancing group operator is responsible for following a planned schedule with a 15-minute resolution. Deviations from the planned schedule are balanced physically by the TSOs and settled financially with the BK. There is a legal obligation on Bilanzkreise to balance their positions to the best of their ability.

Prices for balacning energy are unified across TSOs and determined according to the  balancing energy price settlement system (BK6-12-024) developed by Federal Network Agency (FNA) as of 01/12/2012.

\begin{equation}
reBAP = \frac{\sum net imbalance energy cost}{\sum net imbalance energy volume}
\end{equation}