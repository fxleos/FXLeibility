\chapter{Power Markets and The Role of Flexibility Management}
\label{ch:market}
\textit{This chapter introduces some key concepts of power market elements and how the role of flexibility management is determined by them. We adopted a generalized method to extract the key attributes of power market structures that have impacts on value of flexibility management. The purpose of this chapter is to offer a qualitative analytic framework and}
%a comprehensive and comparative view on flexibility management in different power market regimes.}

\section{Power market frameworks: a comparative analysis}

Started in the 1980s and facilitated in 1990s, liberalized power markets has been the mainstream worldwide, especially in developed countries where the constructions of power infrastructure have been largely completed \cite{Srivastava2011,Ranci2013}.
Nowadays, there are many maturely existing liberalized power markets. However, since different preconditions exists in different countries due to historical, political and climatical reasons, the structure of their power markets tend to be very heterogeneous. Moreover, with the development of technologies, for instance the renewable penetration and rise of demand response as well as electricity storage, power markets face pending or undergoing restructuring, make them a rapidly changing field of the economy. \cite{Ziel2015}

These spatial and temporal variances bring great challenges to our study as the business models of flexiblity managment and values out of them depend extensively on the power market structure. Hereby we reviewed and analyzed the existing mechanisms of how power makets can possibly enable the value creation of flexibility management. Proposing novel market mechanisms is out of the scope of our study. 

%\subsection{General structure of power markets}

%\subsection{Key attributes of power market structure}


\section{Proposed business model}
Power exchange / Power pool

Capacity or not

Locational pricing or not
\subsection{Offering}

\subsection{Customer}

\subsection{Application}


Power exchange vs. power pool

The power exchange is a centralized market that usually uses simple offers/bids in the form of price-quantity pairs. The market operator (MO) solves a convex linear programming (LP) problem on an hour-by-hour basis to match the supply offers with the demand bids and determines the market clearing price without taking into account any technical aspects constraining the operation of the generating units or the transmission system. In a power exchange, the role of system operator (SO) is limited to preserving the system security, while each producer is responsible for self-scheduling his own units by solving a decentralized price-based unit commitment (PBUC).

On the other hand, a power pool is a centrally organized market that usually uses complex unit offers and the independent system operator (ISO, who usually acts both as MO and SO) solves a non-convex optimization problem, where a simultaneous 24-hour co-optimization of energy and reserve resources is performed under a large set of unit and system constraints (e.g., unit start-up and shut-down procedures, minimum-up/down time constraints, min/max power output restrictions, ramp-rate limits, transmission limits etc.). In a power pool, the objective of the ISO is the minimization of the system total production cost through a centralized unit commitment. The producers must follow the commitment schedule and the dispatch instructions issued by the ISO in exchange of the make-whole side payments that guarantee that they will fully recover their total operating costs.

%\section{Power market design and structure}
%\subsection{PJM}

%\subsection{Germany}

%\subsection{Australia}
%%\subsection{Day-ahead energy market}

%\subsection{Real-time balancing energy market}

%\subsection{Regulation market}

%\subsection{Synchronized, non-synchronized and supplementary reserves market}

%\subsection{Overview of other non-market services}

%\section{Power market in Germany}
%\subsection{Day-ahead energy market}

%\subsection{Intraday balancing energy market}

%\subsection{Primary, secondary and tertiary frequency control markets}

%\subsection{Overview of other non-market services}

%\section{Power market in  Australia}
%\subsection{Energy market}

%\subsection{Frequency control ancillary services market}

%\subsection{Overview of other non-market services}

%\chapter{Study on US-PJM}

\section{Market}
In markets with capacity obligations such as PJM, all resources that have a commitment must submit an offer in the day-ahead market.
\subsection{Overview}
\subsection{Market participants}
Market buyer
Market seller: 
Load serving entities: can be buyer or seller as described above
Curtailment service provider:

\subsection{Energy Market}
Day-ahead:
The Day-ahead Market is a forward market in which hourly clearing prices are calculated for each hour of the next operating day based on generation offers, demand bids, Increment offers, Decrement bids and bilateral transaction schedules submitted into the Day- ahead Market.

Real-time balancing market:
The balancing market is the real-time energy market in which the clearing prices are calculated every five minutes based on the actual system operations security-constrained economic dispatch.

Separate accounting settlements are performed for each market, the Day- ahead Market settlement is based on scheduled hourly quantities and on day-ahead hourly prices, the balancing settlement is based on actual hourly (integrated) quantity deviations from day-ahead scheduled quantities and on real-time prices integrated over the hour. The day- ahead price calculations and the balancing (real-time) price calculations are based on the concept of Locational Marginal Pricing.

\subsection{Ancillary Service Market}
Battery storage can participate in PJM?s frequency regulation market, but in Jan. 2017 participation rules changed. PJM altered the benefits factor between the two types of frequency regulation it allows: RegA9 and RegD10 resources. In this adjustment PJM did two things that directly affect energy storage:
§ RegD resources are no longer only used for short-duration (a few minutes) service provision, requiring battery storage to draw power from the grid for longer durations.
§ RegD resources procured during morning and evening ramp times are capped at no more than 26.2\% of the regulation procurement requirement.
Under the new rules, batteries participating in PJM's frequency market as RegD resources are being asked to operate in a longer-duration manner that they were not designed for, which has significant impacts on the economics of these projects. 
%SEPA_2017_Utility_Energy_Storage_Market_Snapshot.pdf
%
\section{Business case}

\begin{itemize}
	\item Day-ahead only
	\item Energy only (Day-ahead + Real-time)
	\item Regulation
	\item Synchronized reserve
	\item Non-synchronized reserve
	\item Day-ahead scheduling reserve (30-min supplementary reserve)
	\item Service mutualization (Day-ahead + Real-time + Regulation/SR/NR + DASR)
\end{itemize}

\section{Model}

\section{Implementation}

\subsection{Optimization}

\subsection{Data}

\section{Result and discussion}

\chapter{AU-NSW}

The NEM consists of five interconnected regions, with the dispatch process centrally managed by the market operator AEMO. Wholesale Regional Reference Prices (RRP) are calculated for each region and set the settlement price for all generators in the region. All transaction is the NEM are settled against a half-hourly spot price. However, dispatch within the NEM is optimised by the operator on 5 min intervals, and as such is considered a ‘fast market’. Fast markets (with short dispatch intervals) provide incentives for dispatchable, flexible capacity rather which would otherwise be met by regulation reserves.
%Riesz J, Gilmore J, Hindsberger M. Market design for the integration of variable generation. In: Evolution of global electricity markets. Elsevier; 2013. p. 757–89.
This is reflected in the relative small size of the Frequency Control Ancillary Services (FCAS) mar- ket relative to wholesale spot market. In 2014, payments through the FCAS market (regulation and contingency) totalled \$30 million, while payments through the spot market totalled \$10.8 billion.
%Estimating the value of electricity storage in an energy-only wholesale market


%\section{Regulatory and market framework for flexibility resourses}