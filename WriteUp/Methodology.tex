\chapter{Methodology for valuation of selected flexibility technologies in selected markets}

\section{Modular approach to build valuation models}

\section{Market-based modules}

\subsection{Revenue module}

As is defined in the scope, only explicit revenues from power markets are accounted in our study. At each time step ($t \in T$), the revenue is calculate as the  amount of energy ($e$, in MWh) offered in each energy market segment ($i \in I$), and/or amount of reserve ($r$, in MW) offered in each reserve market segment ($j \in J$), mutiplied by their corresponding prices ($\pi$, in \$/MWh or \$/MW). In reserve market, there are additional revenues from energy provision while the committed capacities are activated by the system operators. The amounts of energy delivered in reserve market are computed using a ratio ($\delta$, in MWh/MW). Equation \ref{eq:module-revenue} illustrates how the overal revenue is determined. 

\begin{equation}
\label{eq:module-revenue}
Revenue = \sum_{t}^{t \in T} \left( \sum_{i}^{i \in I}  \pi_t^{e,i} (e_t^{d,i} - e_t^{c,i})  + \sum_{j}^{j \in J} (\pi_t^{e,r,j} \delta_t^{j} + \pi_t^{r,j}) r_t^j \right)
\end{equation}

where, $d$ and $c$ in the superscripts denote "discharge" (to release energy from flexiblity resrouces to grids) and "charge" (to intake energy from girds to flexiblity resrouces) respectively. $e_t^{d,i}$, $e_t^{c,i}$, $r_t^{j}$, are decision variables.

$I$ and $J$ are subsets of the selected market segments, which vary from region to region depending on their market structure.

The superset of I is the set of selected energy market segments in different geographies:

\begin{equation*}
I \subseteq  \begin{cases}
\{Day~Ahead, Real~Time\} & PJM \\
\{Day~Ahead, Intraday, Balancing\} & Germany \\
\{Real~Time\} & NSW
\end{cases}
\end{equation*}

The superset of I is the set of selected reserve market segments in different geographies:

\begin{equation*}
J \subseteq  \begin{cases}
\{RegA, RegD, SR, NSR, DASR\} & PJM \\
\{PCR, SCR+, SCR-, TCR+, TCR-\} & Germany \\
\{Lower, Raise\} \times \{REG, 6SEC, 60SEC, 5MIN\} & NSW
\end{cases}
\end{equation*}

$I$ and $J$ are sliced according to the business case being studied. For example, we can set $I = \{Day~ahead\}$ and $J=\emptyset$ in order to value the offerings in day-ahead market of PJM. If there are multiple elements in $I \cup J$, it means the flexiblity resource can be reallocated to make offers to different market segments. In these cases, additional market contraits will be required in avoidance of violating actual market rules.

The ratios $\delta$ are computed from the real data, as the system average ratios using the total capacity ($\hat{e}_t^{r,j}$) and total activated energy ($\hat{e}_t^{r,j}$) at each time step.

\begin{equation*}
\delta_t^j = \frac{\hat{e}_t^{r,j}}{\hat{r}_t^j}
\end{equation*}

Price signals, $\pi_t^{e,i}, i \in \{Day~Ahead, Real~Time\}$, can be obtained eith directly from the datasets or from the outputs of the market simulation module described in proceeding section.

$\pi_t^{e,i}, i \in \{Balancing\}$, is the the price for balancing energy (reBAP), which exist only in Germany and has been introduced in Section \ref{market:germany}. It is also available to be retrieved from datasets directly.

Determination of prices in reserve markets, $\pi_t^{r,j}$ and $\pi_t^{e,r,j}$ varies between market geopraphies. While the general rules have be discussed in Chapter \ref{ch:market}, hereby we will illustrate the formulations and calculations of $\pi_t^{r,j}$ and $\pi_t^{e,r,j}$, mathematically and respectively for each market.

\textbf{PJM:}

The real-time market price is applied for all deviations from day-ahead planned schedule, including Regulation, Primary and Supplementary Reserves.

\begin{equation*}
\pi_t^{e,r,j} = \pi_t^{e,i} ~~~~ i \in \{Real~Time\}, j \in \{RegD, RegA, SR, NSR, DASR\}
\end{equation*}

The capacity prices of reserves are computed using a complex algorithm, taking into account a list of specifications of the resrouce, e.g. the performance \& historical performance, benefits factor, milleage, etc. The detailed calculations can be found in appendix. As outputs, we will get deterministic values for $j \in \{RegA, SR, NSR, DASR\}$, and the upper and lower bounds, $\overline{\pi}_t^{r,j}$ and $\underline{\pi}_t^{r,j}$, for $i \in \{RegD\}$.

\textbf{Germany:}

$\pi_t^{r,j}$ and $\pi_t^{e,r,j}$ are based on principle of pay-as-bid. The weighted-average values are available in the datasets.

\textbf{Australia:}

The unit prices of reserve products, $\pi_t^{r,j}$ and $\pi_t^{e,r,j}$, are not available in datasets published by AEMO. Only weekly summary for total payment and recovery are provided. Due to the limits of available data, we are only able to perform calculations of total potential revenues, rather than thorough studies as in the other two geopraphies.

~\newline

Since the revenue module will be used in optimizations, we re-formulate it as following:

\begin{equation*}
Revenue = \textit{\textbf{f}}~X
\end{equation*}

where $X$ is the vector for all desicion variables. For certain sets of market segments $I$ and $J$, $X$ can be derived using Equation \eqref{eq:decision-variable-1} $\sim$ \eqref{eq:decision-variable-end} with $i \in I$ and $j \in J$.
\begin{equation}
\label{eq:decision-variable-1}
X =
\begin{bmatrix}
E^d \\ E^c \\ R
\end{bmatrix}
\end{equation}

\begin{equation}
E^d =
\begin{bmatrix}
E^{d,I(1)} \\ \vdots \\ E^{d,i} \\ \vdots \\ E^{d,I(|I|)}
\end{bmatrix} \\
E^{d,i} = 
\begin{bmatrix}
e_1^{d,i}~e_2^{d,i}~\dots~e_T^{d,i}
\end{bmatrix}^T
\end{equation}

\begin{equation}
E^c =
\begin{bmatrix}
E^{c,I(1)} \\ \vdots \\ E^{c,i} \\ \vdots\\ E^{c,I(|I|)}
\end{bmatrix} \\
E^{c,i} = 
\begin{bmatrix}
e_1^{c,i}~e_2^{c,i}~\dots~e_T^{c,i}
\end{bmatrix}^T
\end{equation}

\begin{equation}
\label{eq:decision-variable-end}
R =
\begin{bmatrix}
R^{J(1)} \\ \vdots \\R^{j} \\ \vdots \\ R^{J(|J|)}
\end{bmatrix} ~~~\\
R^{j} = 
\begin{bmatrix}
r_1^{j}~r_2^{j}~\dots~r_T^{j}
\end{bmatrix}^T
\end{equation}

Function $\textbf{f}$ can be obtained analogously using Eqution \eqref{eq:decision-f-revenue-1} $\sim$ \eqref{eq:decision-f-revenue-end} with $i \in I$ and $j \in J$.
\begin{equation}
\label{eq:decision-f-revenue-1}
\textit{\textbf{f}} =
\begin{bmatrix}
\Pi^e~|~&-\Pi^e~|~&\Pi^{e,r} \Delta + \Pi^r
\end{bmatrix}
\end{equation}

\begin{equation}
\Pi^e =
\begin{bmatrix}
\Pi^{e,I(1)}~|~&\dots~|~&\Pi^{e,I(|I|)}
\end{bmatrix} ~~~\\
\Pi^{e,i} = 
\begin{bmatrix}
\pi_1^{e,i}~\pi_2^{e,i}~\dots~\pi_T^{e,i}
\end{bmatrix}
\end{equation}

\begin{equation}
\Pi^{e,r} =
\begin{bmatrix}
\Pi^{e,r,J(1)}~|~&\dots~|~&\Pi^{e,r,J(|J|)}
\end{bmatrix} ~~~\\
\Pi^{e,r,j} = 
\begin{bmatrix}
\pi_1^{e,r,j}~\pi_2^{e,r,j}~\dots~\pi_T^{e,r,j}
\end{bmatrix}
\end{equation}

\begin{equation}
\Pi^r =
\begin{bmatrix}
\Pi^{r,J(1)}~|~&\dots~|~&\Pi^{r,J(|J|)}
\end{bmatrix} ~~~\\
\Pi^{r,j} = 
\begin{bmatrix}
\pi_1^{r,j}~\pi_2^{r,j}~\dots~\pi_T^{r,j}
\end{bmatrix}
\end{equation}

\begin{equation}
\label{eq:decision-f-revenue-end}
\Delta = diag (
\delta_1^{J(1)}, \dots , \delta_T^{J(1)}, \dots, \delta_1^{J(|J|)}, \dots, \delta_T^{J(|J|)})
\end{equation}



\subsection{Market simulation module}

The revenue from arbitrage depends extensively on the movement of the price, which can be influence by factors including the activities of arbitrage themselves. 



\subsection{Market constraints}

Energy constraints:

Day-ahead
\begin{equation}
\hat{e}_t^i - \hat{e}_t^{peak} \leq e_t^{d,i} - e_t^{c,i} \leq \hat{e}_t^i - \hat{e}_t^{base} ~~~ i \in \{Day~Ahead\}
\end{equation}

Real-time



Capacity constraints:
\begin{equation}
r_t^j \leq \hat{r}_t^j
\end{equation}


\section{Technology-based modules}

\subsection{Cost module}

\subsection{Technology simulation module}

\subsubsection{Energy Storage}

\begin{equation}
\label{eq:tech-ESS}
s_t = \eta_s s_{t-1} + \eta_c (\sum_{i}^{i \in I} e_t^{c,i} + \sum_{j}^{j \in J}\delta_t^{j,-}r_t^j)- \frac{1}{\eta_d} (\sum_{i}^{i \in I} e_t^{d,i} + \sum_{j}^{j \in J}\delta_t^{j,+}r_t^j)
\end{equation} 

where, the energy to reserve ratios are separated to positive and negative components:

\begin{equation}
\label{eq:ratio-pos}
\delta_t^{j,+} = \begin{cases}
\delta_t^j & \delta_t^j  \geq 0\\
0 & \delta_t^j  < 0
\end{cases}
\end{equation}
\begin{equation}
\label{eq:ratio-neg}
\delta_t^{j,-} = \begin{cases}
0 & \delta_t^j  \geq 0\\
-\delta_t^j & \delta_t^j  < 0
\end{cases}
\end{equation}
\newline
In order to formulate Equation \eqref{eq:tech-ESS}, we first introduce a matrix denoted $M$:
\[
M
=
\begin{bmatrix}
\eta_s^0 & 0 & 0 &  \dots & 0 \\
\eta_s^1 & \eta_s^0 & 0 &  \dots & 0 \\
\eta_s^2 & \eta_s^1 & \eta_s^0 &  \dots & 0 \\
\vdots & \vdots & \vdots &  \ddots & \vdots \\
\eta_s^{T-1} & \eta_s^{T-2} & \eta_s^{T-3} & \dots & \eta_s^0 \\
\end{bmatrix}
\]
\newline
Then $M$ is used to construct $M^I$ and $M^J$ with a given pair of sets of market segments $I$ and $J$.
\begin{equation*}
M^I = \begin{bmatrix}
M^{I(1)}~~|&\dots~~|&M^i~~|&\dots~~|&M^{I(|I|)}
\end{bmatrix}~~~
M^i = M ~~~ \forall i \in I
\end{equation*}
\begin{equation*}
M^J = \begin{bmatrix}
M^{J(1)}~~|&\dots~~|&M^j~~|&\dots~~|&M^{J(|J|)}
\end{bmatrix}~~~
M^j = M ~~~ \forall j \in J
\end{equation*}
\newline
Similar to Equation \eqref{eq:decision-f-revenue-end}, we reconstruct the diagonal matrices with the decomposed ratios from Equation \eqref{eq:ratio-pos} and \eqref{eq:ratio-neg}.
\begin{equation*}
\Delta^+ = diag (
\delta_1^{J(1),+}, \dots , \delta_T^{J(1),+}, \dots, \delta_1^{J(|J|),+}, \dots, \delta_T^{J(|J|),+})
\end{equation*}
\begin{equation*}
\Delta^- = diag (
\delta_1^{J(1),-}, \dots , \delta_T^{J(1),-}, \dots, \delta_1^{J(|J|),-}, \dots, \delta_T^{J(|J|),-})
\end{equation*}
\newline
Finally, we can derive the matrix form of Equation \ref{eq:tech-ESS}.
\begin{equation}
S = \eta_s M S_0 + \begin{bmatrix}
-\frac{1}{\eta_d} M^I~~|& \eta_c M^I~~|& M^J (-\frac{1}{\eta_d} {\Delta^{+}}^T + \eta_c {\Delta^{-}}^T)
\end{bmatrix} X
\end{equation}
where, $S$ and $S_0$ are vectors for the temporal and initial state, respectively.
\begin{equation*}
S = \begin{bmatrix}
s_1~~s_2~~\dots~~s_T
\end{bmatrix}^T
\end{equation*}
\begin{equation*}
S_0 = \begin{bmatrix}
s_0~~s_0~~\dots~~s_0
\end{bmatrix}^T
\end{equation*}
\subsubsection{Electric Vehicle}
Electric vehicle to grid systems are fundamentally battery energy storage systems in term of their physical dynamics. Therefore, they can be modeled generally using the same approach as in preceding paragraphs. However, there are several attributes that uniquely characterize electric vehicle to grid systems compared to normal battery storages:

\begin{itemize}
	\item The availability of an EV2G system, in terms of delivering both energy (in MWh) and capacity reserve (in MW), is dynamic rather than static, since the number of EVs connected in the power grid is changing all the time with the behaviors of plug-in/ plug-out.
	\item The energy stored in the system will be consumed not only for delivering our targeted services (arbitrage or balancing), but also for driving of EVs themselves. This part of costs will be implicitly captured by the revenue module using Equation \eqref{eq:module-revenue}, which will distort the real value of services provided for the grid. 
\end{itemize}

Therefore, two main modifications are made to adapt the model of ESSs for better representing the EV2G systems: 

\begin{enumerate}
	\item The EV2G system is modeled as a dynamic ESS by taking into consideration the connection/ disconnection of EVs to/ from the grids.
	\item The costs of energy consumed for driving are accounted, following the original plan, i.e. without controlling algorithm for grid services, and added back to the revenue in Equation
\end{enumerate}

In order to implement the first measure, we introduce additional terms to represent the number of EVs entering ($n_t^+$), leaving ($n_t^1$) and remain in ($n_t$) the system at each time step. 
\begin{equation}
n_t = n_{t-1} + n_t^+ - n_t^-
\end{equation}

Thereby the state equation for an EV2G system is written as:
\begin{equation}
\label{eq:tech-EV}
\begin{aligned}
s_t = & \eta_s s_{t-1} + \eta_c (\sum_{i}^{i \in I} e_t^{c,i} + \sum_{j}^{j \in J}\delta_t^{j,-}r_t^j)- \frac{1}{\eta_d} (\sum_{i}^{i \in I} e_t^{d,i} + \sum_{j}^{j \in J}\delta_t^{j,+}r_t^j) \\
&+ s^+ n_t^+ - s^- n_t^-
\end{aligned}
\end{equation} 



