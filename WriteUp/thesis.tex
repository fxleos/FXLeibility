% LaTeX script for Semester und Diploma Thesis
% Martin Geidl, April 2004
% Edited by Theodor Borsche, April 2015
% This template can be compiled with pdflatex or traditional latex
% If used with pdflatex all images have to be in pdf format

\documentclass[a4paper,11pt,twoside,onecolumn]{book}

% encoding of the input file. If your computer/ editor uses utf8, change the option
\usepackage[latin1]{inputenc}

% language packages for hyphenation, etc
\usepackage[german,english]{babel}
\usepackage{ae} % some additional fonts / symbols


% packages for MATH input
\usepackage{amstext,amssymb,amsbsy,amsmath}
\interdisplaylinepenalty=2500
\usepackage{mathtools} % additional symbols and fixes


% packages for TABLES
%\usepackage{rotating}
\usepackage{booktabs} % for tables, use booktabs. See documentation.
\usepackage{floatrow}
\floatsetup[table]{capposition=top} % ensure caption is above table
\usepackage{lscape} % allows to rotate tables
\usepackage{multirow} % allows to use cells spanning multiple rows

\usepackage{url} % allows to typeset url's with line-breaks
\usepackage[ruled,vlined]{algorithm2e} % for typesetting algorithms


% for UNITS, use siunitx
\usepackage[detect-all, per-mode = fraction]{siunitx}


% GRAPHICS packages
\usepackage{graphicx} % for including figures with \includegraphics[graphicx keys]{file}
\graphicspath{{Figures/}} % you can put figures in this path
\DeclareGraphicsExtensions{.pdf,.png,.jpeg,.eps} % you do not need to type these extensions for figure files
% \usepackage{epstopdf} % convert eps to pdf. Requires ghostscript

\usepackage{float} % additional float definitions and control


% adjust FLOAT placement
\renewcommand{\textfraction}{0.15} % minimum amount of text on a page
\renewcommand{\topfraction}{0.85} % maximum amount a figure at top may use
\renewcommand{\bottomfraction}{0.6} % maximum amount a figure at bottom may use
\renewcommand{\floatpagefraction}{0.55} % minimum amount a figure on its own page must use
\setcounter{totalnumber}{4} % maximum number of figures per page
\setcounter{topnumber}{3} % maximum number of figures at top
\setcounter{bottomnumber}{2} % amximum number of figures at bottom

% COLOR
\usepackage[dvipsnames]{xcolor} % xcolor allows to define colors
% the 10 standard ETH colors in CMYK colorspace
\definecolorset{cmyk}{}{}{%
    eth1, 1,    0.7,  0,   0.30;%
    eth2, 0.75, 0.4,  1,   0.40;%
    eth3, 1,    0.5,  0,   0;%
    eth4, 0.3,  0,    1,   0.55;%
    eth5, 0.2,  1,    0,   0.20;%
    eth6, 0,    0,    0,   0.7;%
    eth7, 0,    0.9,  0.8, 0.20;%
    eth8, 1,    0.25, 0.3, 0.1;%
    eth9, 0,    0.55, 1,   0.40;%
    eth10,0.6,  0,    1,   0 }

% RGB colors don't look as good, use CMYK if possible
\definecolorset{RGB}{}{}{%
    eth1rgb,  31,  64, 122;%
    eth2rgb,  72,  90,  44;%
    eth3rgb,  18, 105, 176;%
    eth4rgb, 114, 121,  28;%
    eth5rgb, 145,   5, 106;%
    eth6rgb, 111, 111, 100;%
    eth7rgb, 168,  50,  45;%
    eth8rgb,   0, 122, 150;%
    eth9rgb, 149,  96,  19;%
    eth10rgb,140, 182,  60} % define ETH corporate design colors. Names are 'eth3', ..., 'eth9'
\definecolor{grey}{rgb}{0.9,0.9,0.9}


\selectlanguage{english} % choose language for hyphenation and dates

% \makeatletter
% \newcommand*{\rom}[1]{\expandafter\@slowromancap\romannumeral #1@}
% \makeatother

\usepackage{array}
\newcolumntype{L}[1]{>{\raggedright\let\newline\\\arraybackslash\hspace{0pt}}m{#1}}
\newcolumntype{C}[1]{>{\centering\let\newline\\\arraybackslash\hspace{0pt}}m{#1}}
\newcolumntype{R}[1]{>{\raggedleft\let\newline\\\arraybackslash\hspace{0pt}}m{#1}}

% correct bad hyphenation here
\hyphenation{op-tical net-works semi-conduc-tor}

% Combine citation
\usepackage{cite}

\begin{document}
\frontmatter

% Enter your data in the file titlepage.tex
\begin{titlepage}
\begin{center}

\includegraphics[height=12mm]{Figures/eth_logo_lang_pos}
\hfill
\includegraphics[height=8mm]{Figures/PSL_logo}
\hfill
\includegraphics[height=15mm]{Figures/L+G_logo}

\vspace{30mm} Xingliang Fang \\
\vspace{10mm} \textbf{\LARGE Valuation of Energy Flexibility Solutions in Different Power Market Regimes} \\
\vspace{10mm} Master Thesis \\ PSL1726


\vfill

EEH -- Power Systems Laboratory, ETH Zurich \\
Corporate Strategy Office , Landis+Gyr

\vspace{5mm}

Examiner: Prof.~Dr.~Gabriela Hug \\
Supervisor: Dr.~Donnacha Daly (Landis+Gyr), Jun Xing Chin


\vspace{5mm} Zurich, \today

\end{center}
\end{titlepage}


\chapter*{Abstract}
%\input{abstract}

With penetration of renewable energy resources and development of emerging technologies, the conventional power market regimes for managing and operating flexibility in power systems are being challenged. Regulators and power market designers are constantly revisiting and making changes to the rules, which leads to a highly disruptive business environment for players in power markets, especially for those whose scopes of business are across several power market jurisdictions.

This paper is therefore designed to solve this challenge for power market players by developing a framework for analysis and valuation of flexibility solutions in various power market regimes.

In this thesis, %we first mapped both the academic research and the real-world market regimes to provide a comprehensive and structured view on the topic. Based on that, 
we developed an analytical framework for qualitative analysis and a techno-economic model for quantitative valuation of flexibility solutions. The techno-economic model is built with a modular approach and is adaptive for various market regimes and several technologies. 

Furthermore, we carried out case studies in three power market jurisdictions, i.e. PJM Interconnection, Germany power market, and New South Wales in Australia's National Electricity Market. It is found that these three markets have heterogeneous structures and are at different stages in terms of implementing frameworks for emerging flexibility solutions. In the quantitative studies, it is noticed that except for explicit market rules, there are implicit barriers that may be unfavorable for some flexibility solutions, making the markets not fully open for them. Further to specific technologies, the profitability of two types of flexibility solutions, i.e. battery energy storage systems and electric vehicle to grid was studied.  Results show that batteries are still costly and not profitable in the near future even with drastic cost reduction. However, in case where the costs for batteries are not responsible by market players, such as electric vehicle to grid technology, positive profitability is seen. With the number of electric vehicle growing rapidly, it reveals a promising business area. Finally, we investigated the impacts of renewable penetration, which are found to depend on power market regimes as well.

This thesis is designed to support strategic business planning, primarily for technology vendor but also for all market players that are interested in flexibility solutions with a cross-regional perspective.


\newpage


\chapter*{Acknowledgements}
%\input{preface}
This report summarizes the outcome of the master thesis in cooperation between Power Systems Laboratory (PSL) at ETH and and Corporate Strategy Office (CSO) at Landis+Gyr. 

First of all, I would like to thank Professor Dr. Gabriela Hug, the head of PSL for facilitating this cooperation, as well as Mr. Roger Amhof, the head of CSO for offering me the opportunity working with an intellectual team.   

My deepest gratitude goes to my tutors, Dr. Donnacha Daly and Jun Xing Chin. Thank you both indeed for all the time and effort spent in guiding and tutoring me throughout the whole project. 

Meanwhile, I am also profoundly grateful for Ifigeneia Stefanidou, Felipe De Montagut, and Mathias Haug at Landis+Gyr, who helped me greatly in the project by providing me inputs from a business-perspective. 

Last but not least, I sincerely appreciate my family and friends, as well as colleagues at ETHz for their supports which are also essential to the completion of this work.

\newpage

\tableofcontents
\newpage

%\chapter*{List of Acronyms}
%\addcontentsline{toc}{chapter}{List of Acronyms}

\newpage



\mainmatter




\chapter{Introduction to Flexibility Management and The Goal of This Thesis}
\chaptermark{Introduction}
%http://www.tex.ac.uk/FAQ-runheadtoobig.html
\label{ch:introduction}

\section{Defining flexibility and flexibility management}
\sectionmark{Definitions}
Maintaining balance between supply and demand is a fundamental requirement to electric power system operations. The capability of a power system to match the supply and demand at each point of time by using control resources are often referred to as ``operational flexibility", or simply ``flexibility" \cite{Cochran2014,Wang2017,Lund2015,Delft}. Flexibility is therefore not a new concept. Power systems are inherently with uncertainty and variability since loads vary over time and occasionally in unexpected ways, and power plants may suffer unpredictable failures sometimes. All power systems are designed and built with certain level of flexibility to cope with those unexpected events. Conventionally, the flexibility is mainly enabled on the supply side, where dispatchable resources are controlled to adjust their outputs to match the time-varying load.

However, following radical transformations towards decarbonization, decentralization and digitalization in the energy industry, the existing operating model of electricity flexibility is being critically challenged and increasing interests are moving to flexibility from the load side and energy storage technologies\cite{Lund2015,Bronski2015,McKinsey&Company2010}. These disruptions are not only in technological but also in institutional and managerial manners, which are sparking market restructures and business model innovations. For instance, those new resources are typically smaller in scale compared the traditional flexible generations so the new operating model shall be managed via more decentralized approaches. Flexibility management, as an emerging business term, refers to the process how those new small-to-medium scale sources of flexibility are enabled, organized and exploited to serve the needs of power systems.

\section{Challenges in power system flexibility}
\sectionmark{Challenges}
The penetration of renewable energy sources (RES), which is spreading over the whole industry globally \cite{Agency2016}, are commonly viewed as the fundamental driver for the transformation that is creating operational challenges in maintaining system balance with existing flexibility resources \cite{Cochran2014,Wang2017,Lund2015,FraunhoferIWES2015,Muller2016,Kwon2014,Kondziella2016,Papaefthymiou2016,Alizadeh2016,Bertsch2016}. The impacts of RES on electric power systems can be deduced from the instinct technological attributes of RES \cite{Kondziella2016,Edenhofer2013}:
\begin{itemize}
	\item RES is variable and often viewed as non-dispatchable since its output is determined by weather conditions, and furthermore
	\item RES is usually imperfectly predicted and specific power generation is uncertain until realization.
\end{itemize}

Effects of the property being non-dispatchable can be illustrated by introducing the concept of ``net load", also referred to as ``residual load", which equals the total system load minus the renewable generation so represents the load that needs to be served by non-RES resources\cite{Cochran2014,Muller2016,Ueckerdt2015}.

\begin{figure}[h!]
	\centering
	\includegraphics[width=0.9\linewidth]{Figures/NetLoad}
	\caption{An illustrative example of net load profile \cite{Cochran2014}}
	\label{fig:net-load}
\end{figure}

Figure \ref{fig:net-load} shows an example profile of net load, based on which we can see how RES is stressing the existing non-RES generations:

\begin{itemize}
	\item \textbf{Shorter peaks}: resulting in fewer operating hours for conventional peak generators, affecting their cost recovery and consequently long-term security of supply,
	\item \textbf{Lower turn-down}: diminishing the base load which was stable at a higher without RES, creating challenges to base generators who have limit operational flexibility to vary their outputs, and
	\item \textbf{Steeper ramps}: demanding higher performance in delivering flexibility, eliminating relative low-grade resources from serving the needs for flexibility.
\end{itemize}

It can be seen that the whole span of current generation portfolio serving base, flexible and peak power is under great pressure with the RES growth.

The issue of the forecast error, on the other hand, requires the dispatch of flexibility close to real-time operation. This is especially an explicit issue in places where those activities are organized through power markets. In present power markets, the major part of the scheduling and pre-dispatching is determined ahead of the operating day based on forecasts and errors deviated in real time from the schedule are mostly depending on imbalance settlements via so-call frequency control ancillary services which are typically more costly\cite{Ranci2013,Srivastava2011}. The intra-day market with higher resolution of price signals and shorter prediction horizon toward actual operation is a feasible option and implemented in many markets\cite{Srivastava2011} but intra-day markets are empirically prone to low liquidity in may regions \cite{Lund2015, Hagemann2015,Weber2010}. Without structural improvements in the market design, the demands for frequency control services would raise significantly and thus add burdens to the power system operators \cite{GEEnergyConsulting2014,Krad2017,Koch2009}. Measures such as improving day-ahead forecast \cite{Woo2016}, developing short-term frequency control product \cite{Gonzalez-Aparicio2015}, optimized intra-day \cite{Weber2010} and balancing market frameworks \cite{Wartsila2014}, have been proposed. Being sensitively depending on the market arrangements, existing businesses may be disrupted significantly by any of those market restructures.

Besides, solar power which is forecasted to have even higher potential than wind power in the long run is tending to grow in distributed patterns \cite{Agency2016,Epia2016,Sawyer2016}. With the conventional centralized deployment of flexibility, local congests are likely to exacerbate \cite{Lund2015,STEINKE2013826} which drives the needs for extensions of transmission and distribution capacity.

Collectively, the RES penetrations urge innovations in both technology and market design, which are otherwise burdening power system operators with higher expenses, potentially reducing the revenue scream of existing market players and/ or leading to significant curtailment of RES.

In addition to RES, the electrification of transportation, i.e. the penetration of plug-in electric vehicles (EV), is emerging more recently to be a second pole as the game changer. Facilitated by support policies by states or cities to uncap their multiple benefits such as transport decarbonization, air pollution reduction, and energy efficiency and security, the growth of EV has been accelerating significantly, having exceeded the global threshold of 2 million in 2016 \cite{InternationalEnergyAgency2017}. Although it tends to be treated as a promising source for flexibility by implementing vehicle to grid (V2G) technologies \cite{Size2016,Habib2015,Foley2013}, threats along with it shall not be ignored. Without being fully prepared in technologies and markets, the growth of EV may still move to the opposite side of flexibility with the negative impacts such as increasing peak demand and potential local congestion \cite{Green2011,DBLP:journals/corr/PournarasJZFS17}.

It has been pointed out that lack of flexibility can be identified more intuitively by signals such as \cite{Cochran2014,Wang2017}:
\begin{itemize}
	\item difficulty balancing demand and supply, resulting in frequency excursions or shedded load,
	\item significant renewable energy curtailments,
	\item negative market prices, and
	\item high price volatility in whole power markets.
\end{itemize}

Although having been discussed extensively for years in the academic area and by industry experts, it was not until quite recently when more and more signs of inflexibility had been witnessed did the public start to be indeed aware of the challenges on power system flexibility. For instance, the negative pricing in wholesale power spot market was first introduced in 2007 in Germany intra-day market and in 2008 in Germany/Australia day-ahead market\cite{EPEX_negative_price}, but vast attentions from the public were initiated after 146 hours on 24 days with negative prices were observed in the day-ahead market in 2017. Another famous example could be the power outage in South Australia that happened on September 28th 2016. After a widespread debate, Australia Energy Market Operator (AEMO) finally concluded in its investigation report that the generation deficit of wind farms due to unexpected operations of a control setting responding to multiple disturbances led to the power blackout  \cite{AEMO2016SA}.  This aroused public's worries on supply security from RES generation. As one of the follow-up actions, AEMO partnering with Tesla Inc., one of the leaders in global battery and electric vehicle markets,  built the worlds' largest battery energy storage system (BESS) in South Australia \cite{AEMO_tesla}.

These imply a proper timing for technology vendors to update their assessment on the market, as interests in flexibility management from the public and thus their potential customers have significantly raised.

\section{Technologies: options for system flexibility provision}
\sectionmark{Technologies}

Opportunities on flexibility management are not just being ignited by the increasing level of challenge but also being facilitated by the booming of innovations in technology. Thanks to significant developments in energy storage technologies and information communication technologies (ICT) in recent years, the landscape of flexibility options changed vastly. While it was in the past limited to centralized solutions, extracting flexibility from ubiquitous distributed resources and operating it in an aggregate way has been gradually becoming technically feasible and economical viable.

The availability of technological options to serve system flexibility has shown a great abundance which has never been seen in the past \cite{Cochran2014,Wang2017,Lund2015,Muller2016}. A systematic summary for these various possibilities can be found as Figure \ref{fig:TechnologyOptions}.

\begin{figure}[h!]
	\centering
	\includegraphics[width=0.95\linewidth]{Figures/TechnologyOptions}
	\caption{Overview flexibility options categorized in way of flexibility provision \cite{Muller2016}}
	\label{fig:TechnologyOptions}
\end{figure}

Technologies for flexibility are categorized by their type of provision:

\begin{itemize}
	\item \textbf{Downward-flexibility}: shedding demand or uplifting supply to reduce the positive residual load,
	\item \textbf{Upward-flexibility}: dropping surplus RES feed-in or increasing demand to mitigate negative residual load,
	\item  \textbf{Shifting-flexibility}: shuffling surplus energy from regions (or time steps) with negative or lower residual load to other regions (or time steps) and vice verse.
\end{itemize}

It can be clearly seen that the term demand-side management (DSM), or often referred to as demand response (DR), is actually a granular concept as an umbrella for a list of different technologies with many of them have totally disparate mechanisms. 

Combining the evaluations carried out by several studies \cite{Cochran2014,Wang2017,Lund2015,Muller2016,Despres2017}, the characteristics of different technologies can be summarized on a high level as:

\begin{itemize}
	\item \textbf{Generation}: i.e. flexibility provision by varying power plant outputs. 
	
	This is by far the most mature technology and typically not constrained by the duration of flexibility provision nor how often to be activated. Activation time and ramp rate are the main issues for flexibility from power plants, especially conventional power plants using stream turbines, e.g. coal, lignite and nuclear power plants. Although output adjustments can be done within 1 hour, a cold start may take up to 100 hours or at least 4 hours even by the state-of-the-art thermal power plants \cite{Muller2016,AgoraEnergiewende2017}. Gas turbines are more flexible even compared to some other advanced technologies that are to be introduced later, so they are deemed as a decent option to increase system flexibility \cite{Muller2016}.
	
	Cost is a complex topic and varies greatly between different type of generation technologies but in general they are still lower than most emerging technologies. However, building power plants is not a economical option to cover the extreme events that are rarely seen, as heavy fixed costs of building power plants can be hardly recovered in this scenario. Meanwhile, the issue related to consumption of fossil fuels and green house gas emission raise the uncertainty of operating costs in long term.
	
	\item \textbf{Load shedding}: i.e. load curtailment, mainly enabled by disrupting some energy-intensive industrial processes. In contrast to load shifting, shedded load will not be compensated later on as most of the time the industrial processes are running at their maximum allowances.
	
	Load shedding applications can provide fast responses, but are constrained at duration and numbers of activation. Nonetheless, short timespan of flexibility provision and limited occurrence fit the characteristics of extreme disturbances in power systems, so load shedding can be deployed for that specific purpose.
	
	The activation cost is essentially the loss caused by the disrupted productions so is indeed an adverse factor. The fixed cost, on the other hand, is less concerning as most industry plants nowadays are already equipped with automatic and intelligent energy management systems.
	
	\item \textbf{RES curtailment}: i.e. regulating the outputs of RES plants downwards.
	
	Technically, there are few constraints for RES curtailment as they can be performed promptly and frequently, and last for an indefinite time period. However, since curtailments will waive the revenues that otherwise be received by selling electricity in the market, RES operators are discouraged to do so. Although a list of measures that possible for power system operators to mandate curtailments, it is contradictory to the concept of decarbonization. 
	
	Therefore, we deem the RES curtailment as a compromise and the last option if the needs for flexibility cannot be fulfilled by any other means.
	
	\item \textbf{Power-to-X(P2X)}: i.e. consuming excess electricity to produce other energy carriers, e.g. hydrogen, methane or heat.
	
	P2X technologies can also provide fast response and theoretically last for an indefinite period of time. However, in reality it is virtually constrained by how the by-products are stored and utilized, and values of the by-products also vary significantly in different situations. For instance, while heat generation is valuable in winter, it is likely to be counterproductive in summer. 
	
	Regarding the cost, power-to-gas technologies requires significant high initial investments on equipment while power-to-heat costs much less less with the core components being boilers and heat tanks. Overall, the economics of P2X is still a challenging issue as the value can be harvested only if the by-products are competitive compared to goods by other production methods. However, production of P2X is destined to be intermittent as it would only be activated while upward-flexibility is needed, making it difficult to be economically viable.
	
	\item \textbf{Energy storage}: a system that can absorb surplus energy in time with negative or low residual while release energy in time with higher demand. Due to its technical nature, the energy storage can act on both supply and demand side or be viewed as a third pole \cite{Gunter2016}.
	
	Energy storage itself is a umbrella for an abundance of technologies, including battery energy storage systems (BESS), pumped hydroelectric storage (PHES), compressed air energy storage (CAES), flywheel, thermal storage, and others. These technologies vary significantly in their mechanism and thus in technical parameters such as size and efficiency as well as in performances, e.g. duration, action time, cost, etc. Among them, BESS could be the most attractive with fast response (activated within seconds), decent duration (up to 10 hours) and most importantly few external dependencies, like geographic conditions. Cost is the main concern for batteries, but is decreasing dramatically in recent year \cite{Nykvist2015}.
	
	%\cite{Rastler2010,Eyer2010}
	
	%\begin{itemize}
		%\item \textbf{Batteries} use electrochemical reactions, i.e. oxidation-reduction processes between the battery's electrolyte and electrodes, converting electricity to chemical energy stored and vice verse. Battery is one of the most focused areas in research and an increasing number of novel batteries are being developed. More familiar and mature ones include lead-acid, lithium-ion, sodium/sulfur (Na/S), and others. Flow batteries, sometimes
		%\item \textbf{\item{}}
	%\end{itemize}
	
	\item \textbf{Load shifting}: corresponding to the concept of demand response in a narrower sense where responsive loads are enabled by direct control signals or indirect price signals.
	
	There are a great variety of load types that can be exploited for load shifting, so similar to energy storage load shifting contains a list of subcategories. However, unlike other technologies that can be characterized by relative standard models, load shifting shows a higher diversity. This is because the characteristics of a load shifting system would be sensitively affected not only the technical parameters of load but also the control strategy and the users' preferences. Nonetheless, the load shifting in general has short activation time (within seconds to minutes), short duration (typically 0.5 to 8 hours) and relatively low cost (even close to zero if applications are equipped with control devices). 

	\item \textbf{Electricity grid}: i.e. extension of distribution and transmission capacity. Distinguishing other technologies discussed above that shuffle electricity temporarily, the grid extension is the only option that deals with fluctuations of residual load spatially. 
	
	Flexibility from transmission and distribution (T\&D) network has the fastest response, indefinite duration and unlimited numbers of activation so together with the generation flexibility it has been a main solution for conventional power system flexibility. However, challenges come from the development of distributed energy resources (DER) which are disrupting the existing T\&D systems with altered electricity flow profiles and congestion in the network is a major bottleneck for delivering flexible power in the grid. Further grid infrastructure upgrade is necessary but leads to exceptionally high expenses so should be complemented by other technologies introduced above \cite{Cochran2014}.
	
\end{itemize}

Studies reveal that an abundance of different flexible technologies will be available in the future, and it is well agreed that no single option would be sufficient to provide flexibility to power systems \cite{Cochran2014,Wang2017,Lund2015,Muller2016}. Determining the the best mix of options need to be carried out on a case base and requires significant efforts as being a complex technological-institutional-economic issue. 

The innovations in technology, changes on market frameworks and cost reductions will collectively change the landscape, and overall create more available options for players. Therefore, technology vendors shall keep watching the development of technologies and constantly update their view on which technologies to supply. 

\section{Applications, benefits and business models}
\sectionmark{Applications}
So far we have introduced the mega-trends with growing challenges and opportunities on power system flexibility. It is however not sufficient to understand the value of flexibility from a business-oriented point of view. More detailed analysis on applications, benefits and business models are necessary.

Here application means a use where flexibility is exploited for a certain aim via certain procedures. Benefit denotes a value that can be evaluated in monetary, financial or social terms. Business model describes the rationale of how the application-specific benefits are being captured by a certain business player.


 Generally, these three items are disparate in a liberated power market compared to a vertically integrated market.

\subsection{In liberalized market}


%\subsubsection{Needs of different plyaers}

Player * Market * Application


%\subsubsection{Energy Markets}


%\subsubsection{Ancillary Service Markets}

\subsection{In vertically integrated market}
(placeholder)
\newpage


\section{Scope and research questions}

\subsubsection{Scope of benefits}

Explicit revenue from power markets

\subsubsection{Scope of technology}

Firstly, we are focusing on small-to-medium scale emerging solutions in low-to-medium voltage level, so flexibility provisions from conventional generations and pumped hydro power plants are excluded. Electricity grid extension also falls into this category, plus its value is not accounted as explicit revenue in power markets, so it is also excluded.

Secondly, RES curtailment as is mentioned previously is consider as a compromise rather than opportunity. Moreover, the benefits of RES curtailment may be valued from a system point view for grid stability maintenance. It will usually lead to no increase on explicit revenue for the play from power markets that is of our interests, unless the RES operators are obligate to meet the schedule and published for deviations.

P2X are also excluded, because the values of its by-products such as hydrogen and heat are hard to account in a generic way and definitely not a explicit revenue from the power markets.

Load shedding is not an emerging technology with few opportunities for technology vendors. Besides, they depend on type of load and are not ...

Energy storage and DSM. DSM is too granular as is analyzed. Different specific system has its own dynamic. We take ESS and EV as special example of . 





The target audience of this thesis is the management at Landis+Gyr on a high coporate level.

The ultimate goal is to provide references to support the audiences' strategic decision makings regarding flexibility management.

In order to achieve this, we conducted qualitative studies and developed quantitative models to identify: 1) the value of markets for flexiblity management

\begin{itemize}
	\item 
\end{itemize}

The goal of this thesis is to:

developed a robust modeling tool with moderate complexity so that it can not only provide results in current environment but can be also reused or easily revised to provide results in case of changes in the future.

based on the tool, make quantitative as well as quanlitative analysis to provide refer 

Purpose: providing references for strategic decision makings regarding flexibility management.

In order to make the analysis robust and reliable, we have built a techno-economic models which include the bottom-up dynamics of some key elements regarding the electricity markets and flexilibity technologies. 

However, it shall be noticed this thesis is not intended to serve for:

project developers to design a flexiblity system or make operating (including bidding) strategies of the system

policy makers to redesign the electricity market structure, rules or other policies

grid planners to understand the needs and options of flexibility in order to acheive system relability with lowest costs


Since the concept of flexiblity management is related to a great variety of technologies, applications and Landis+Gyr is positioning globally in various markets, the scope could be very broad. Nonetheless, in order to produce viable and reliable results with a solidily established techno-economic model, we have to make comprises. According to the relevance to Landis+Gyr's business, the scopes are defined as:

%\subsection{Scope of technologies}

%\subsection{Scope of applications}

%\subsection{Scope of benefits and business models}
The potential business model of Landis+Gyr is either to supply products to the customers to help them enable flexibility or to directly sell them flexible MWs as a service. In this case, we want to understand the value of each MW we enabled or sold. We assume Landis+Gyr will not directly partipate and trade in the power market, as it is going to place Landis+Gyr at the rival side of some customers in that market.

The value of flexibility will definitely vary according to the purpose, users' portfolio and operating strategies. 


%\subsection{Scope of markets}
\chapter{Literature Review}
\label{ch:LitRev}


As is clearly revealed by the literuare review, there exist abused research articles generally on this topics of flexibility management. However, there exist very few academic works that serves the needs of our target audiences who are the management of technology vendors. The deviationsof interests result in gaps that make it difficult to directly use the existing works. These gaps include:

\begin{itemize}
	\item Most of the researches are based on one specific technology and one specific market, as usually a utility company or a grid planner is operating in one market regimes and a technical professional is focusing on one technology.  However, our target audiences are likely to be interested in various markets and technologies.
	\item Scope
	\item Mothed - proof of concept
\end{itemize}




Conventionally, their decision makings are supported primarily by commercial consulting firms who relied much on qualitative anlysis or quantitative data-anlytics. Even when sometimes it is possible that those firms have developed model with fundamental and physical approach, the model is always customized and not public 

most of the researches are focusing one specific technology and one specific market, due to the nature of their target audiences. However, the managment iof a technology vendor will likely to be interested in various markets and various technologies. 

The economics of flexibility solutions in power systems, especially electric energy storage (EES), is an active topic in research. It has drawn great attentions from the academics, investors and policy makers. 


% electricity storage are currently in the focus of research, by academics, utilities, potential investors as well as policy makers. The present document is the result of the analysis of more than 200 publications on that subject. It aims at presenting the “state of the art” regarding research on the economics of electricity storage. Three particular aspects are given attention to: the methodologies used, the profitability results obtained and the impact of regulation on storage economics.
\section{Purpose and stakeholder}


\section{Modelling methodology}
%Two broad approaches have been taken to modelling UK electricity systems. As discussed in Section 2.2.1, system studies tend to employ a holistic and system wide perspective with only coarse temporal resolution. Other studies attempt to understand system balancing with high penetration of wind and issues arising from ramp and slew rates of wind and errors in wind forecasting. These studies require high temporal resol- ution and therefore tend to simulate short periods of time and make static assumptions for system context.(Black and Strbac, 2007, 2006; Pelacchi and Poli, 2010; Barton and Infield, 2004; Bathurst and Strbac, 2003)
\subsection{Overview}
Engineering vs system
Linear vs nonlinear
Deterministic vs stochastic problems
Solving techniques

\subsection{Engineering model}
Price taker
perfect forecast
stochastic or dynamic programming
Hybrid system
Service mutualization

\subsection{System model}


\section{Affecting factor}
\subsection{Techno-economic characteristics of power system}
\subsubsection{Generation}
Generation mix (Renewable integration)
Fuel Prices
\subsubsection{Climate and weather}

\subsubsection{Transmission}
Grid topology
Transmission capacity

\subsubsection{Consumption}

\subsubsection{Merit-order model}
\cite{Sensfuss2008}

\cite{SaenzdeMiera2008}
\cite{Tveten2013}
\cite{McConnell2013}
\cite{Gelabert2011}
\cite{Clo2015}
\cite{Woo2016}
\cite{Cludius2014}
\cite{He2013}
\cite{Mulder2013}

\subsection{Statistic model}
\cite{Alipour2017}
%Efficient modelling of 

\subsection{Perfect forecast}
\cite{He2011}
\cite{Sioshansi2009}
\cite{Bathurst2003}
\cite{Drury2011}
\cite{Connolly2012}

\subsection{Power market degisn and policy regulation}
\subsubsection{Player and competitive landscape}

\subsubsection{Renewable Support Scheme}

\subsubsection{Power Market Design}
Market structure and rules: nodal, interval, reserve market
Access

In general, the seven ISOs/RTOs require companies that service loads (i.e., the energy re- quirements of end-use customers) to provide reserves in proportion to their loads. (ref to Project Report: A Survey of Operating Reserve Markets in U.S. ISO/RTO-managed Electric Energy Regions)

Balancing market design \cite{Wartsila2014} \cite{Moller2010}

\subsubsection{Ownership and dispatch}

\subsubsection{Direct policy support}
Capacity market
Feed-in premium or tariff
Other program

\section{Value of results for reference}
\subsection{Demand for flexiblity in power system}

\subsection{Profitability of flexibility solutions}

\chapter{Power Markets and The Role of Flexibility Solutions: An Analytical Framework}
\chaptermark{Market regimes}
\label{ch:market}
\textit{This chapter aims at offering a comparative view on different power market regimes, based on which an analytical framework can be established. Such a framework offers technology vendors a solid foundation for qualitatively analyzing the opportunities of flexibility solutions in a given market context. By mapping a list of mature power markets worldwide, we extract some key attributes of power market structures that impact the value of flexibility solutions.}

\section{Motivation for a power market analysis framework}

Conceived in the 1980s and facilitated in the 1990s, liberalization of power markets has become the mainstream worldwide \cite{Srivastava2011,Ranci2013,Vagliasindi2013}. However, different conditions exist across economies including historical, political and climatic factors. As a result, structures of these power markets tend to be very heterogeneous. %Moreover, with development of technologies, power markets face pending or undergoing restructuring, make them a rapidly changing field of the economy. \cite{Ziel2015}
This brings great challenges to companies that pursue a cross-regional or even global footprint, since business models for flexibility solutions as well as their feasibility and performance depend extensively on the power market structure. Compared to other stakeholders that are interested in flexibility solutions such as utilities and regulators, technology vendors are more likely to have international ambitions. This is not only because they have fewer regulatory barriers, but also because firms with higher research and development (R\&D) intensity have stronger motivation for expanding geographic boundaries to mitigate market risks and seek growth opportunities \cite{Brouthers2007}.

Therefore, in this chapter we map the taxonomy of power markets with a particular focus on characteristics related to flexibility solutions, in order to provide a general framework for technology vendors facing different power markets.

Such a global view is established by generalizing and comparing market regimes in different systems that are listed in Table \ref{tab:markets}. We will name a few of them as typical examples while discussing each structural attribute. However, it shall be noted that the goal of this chapter is not to provide comprehensive analysis on each of the system. With on-going restructuring of market regimes, each system is constantly evolving over time. Taking the electricity market in Great Britain (GB) as an example, it had been operating in the model of power pool for over 10 years before it reformed to a power exchange arrangement in 2001 \cite{Rebours2009,FrontierEconomics2016,ofgem_m}, and in a more recent restructuring in 2014 they established the capacity market that did not exist there before \cite{ofgem_cm}. 

Nevertheless, our general framework will remain largely stable regardless of adjustments in individual markets. This again reveals the importance of an analytical framework facing such a fast-changing area. Using the same example in GB, readers can immediately identify potential opportunities riding on the introduction of capacity market by referring to Section \ref{sec:CM}.


\begin{table}[h!]
	\small
	\centering
	\begin{tabular}{L{5cm} l l l}
		\hline
		\hline
		\textbf{System} & \textbf{Abbreviation} & \textbf{Country} & \textbf{Main Reference} \\
		\hline
		\hline
		PJM Interconnection & PJM & US & \cite{Rebours2009,Srivastava2011,Cochran2013,EllisonJ.F.TesfatsionL.S.LooseV.W.Byrne2012,Gilstrap2015,Brown2015,Borenstein2015,PJM_web,PJM2017b,PJM2017c}\\
		\hline
		New York ISO & NYISO & US & \cite{Cochran2013,EllisonJ.F.TesfatsionL.S.LooseV.W.Byrne2012,Gilstrap2015,Borenstein2015,NYISO_web}\\
		\hline
		Midcontinent ISO\footnote{Formerly named Midwest ISO}& MISO & US \& Canada& \cite{EllisonJ.F.TesfatsionL.S.LooseV.W.Byrne2012,Gilstrap2015,Borenstein2015,MISO_web}\\
		\hline
		ISO New England & ISO-NE & US & \cite{EllisonJ.F.TesfatsionL.S.LooseV.W.Byrne2012,Gilstrap2015,Borenstein2015,ISO_NE_web}\\
		\hline
		California ISO & CAISO & US& \cite{Rebours2009,EllisonJ.F.TesfatsionL.S.LooseV.W.Byrne2012,Gilstrap2015,Borenstein2015,CAISO_web}\\
		\hline
		Southwest Power Pool & SPP & US & \cite{EllisonJ.F.TesfatsionL.S.LooseV.W.Byrne2012,Gilstrap2015,Borenstein2015,SPP_web}\\
		\hline
		Electric Reliability Council of Texas & ERCOT & US & \cite{Srivastava2011,EllisonJ.F.TesfatsionL.S.LooseV.W.Byrne2012,Brown2015,Gilstrap2015,Borenstein2015,ERCOT_web}\\
		\hline
		Ontario Independent Electricity System Operator& IESO/ Ontario& Canada & \cite{Cochran2013,Brown2015,Ontario_web}\\
		\hline
		Alberta Electric System Operator & AESO/Alberta & Canada &  \cite{Brown2015,Alberta_web}\\
		\hline
		National Electricity Market (Australia) & NEM & Australia & \cite{Srivastava2011,Brown2015,AEMO2010,AEMO2015a}\\
		\hline
		National Electricity Market of Singapore & NEMS & Singapore & \cite{Brown2015} \\
		\hline
		Germany\footnote{Referring to territories of 4 TSOs, Tennet, Amprion, 50 Hertz, TransnetBW under the regulation of Bundesnetzagentur (BNetzA) with large volume of electricity traded OTC and on power exchange, EPEX SPOT. } & DE & Germany & \cite{FrontierEconomics2016,Wartsila2014,ConsentecGmbH2014,Deloitte2015} \\
		\hline
		Single Energy Market (Ireland) & SEM/ Ireland & Ireland & \cite{FrontierEconomics2016,Cochran2013}\\
		\hline
		Great Britain\footnote{Referring the territory of the TSO, National Grid, under the regulation of Office of Gas and Electricity Markets (ofgem) with large volume of electricity traded OTC and on power exchange, APX Power UK and N2EX .} & GB & Great Britain & \cite{Rebours2009,FrontierEconomics2016,ofgem_cm,ofgem_m,EnergyUK2017} \\
		\hline
		Other European Markets & - & - & \cite{FrontierEconomics2016}\\
		\hline
		\hline
	\end{tabular}
	\caption{List of markets involved in this study} \label{tab:markets}
\end{table}

In Chapter \ref{ch:introduction}, we identified three applications of flexibility solutions in wholesale markets, including:

\begin{itemize}
	\item \textbf{Arbitrage in energy market}, and
	\item \textbf{Frequency control in ancillary services market}, and
	\item \textbf{Supply adequacy in capacity market}. 
\end{itemize}

Correspondingly, we systemically investigate how the feasibility of these applications is influenced by different market regimes, i.e. structure of energy/ ancillary service/ capacity markets, in the reminder of this chapter. Unlike many other studies that are also focused on comparison of different market structures but for the reference of market designers, we do not analyze the full rationale behind the market design nor their comprehensive merits and drawbacks. Instead, this chapter is focused only on the differences themselves and their direct impacts on value of flexibility solutions.


~\newpage

\section[Flexibility solutions in energy markets]{Flexibility solutions in energy markets%}
	\sectionmark{Energy market}}
\sectionmark{Energy market}
\label{sec:market-energy}
We start our analysis from the wholesale energy market as it constitutes the central transaction platform in power markets \cite{Cochran2013}. 

In a competitive market price should act as an effective signal to coordinate the balance of supply and demand. Reflecting this principle in energy markets, if a market is well-designed, price volatility would increase due to lack of flexibility and in turn become an incentive to encourage participation of new flexibility sources, as introduced in Chapter \ref{ch:introduction}. However, it is not always the case in reality since power market design takes into consideration for not only economic but also physical and political factors. Moreover, since market design is likely to lag behind technological development, some legacy rules tend to create barriers for new technologies even if they may already be favored by those physical, economic and political requirements.

Therefore, although energy arbitrage that can absorb energy in supply surplus and release energy in supply shortage is theoretically beneficial to power systems, it is not always feasible depending on market rules. 

\subsection{Market model: Power pool vs. power exchange}

First of all, it is worthwhile to point out the difference between power pool and power exchange, since they represent two fundamentally distinct approaches of how power markets are organized. 

\begin{figure}[h!]
	\centering
	\includegraphics[width=0.95\linewidth]{Figures/PowerPoolExchange}
	\caption{Illustration of difference between power pool and power exchange}
	\label{fig:pppx}
\end{figure}

As shown in Figure \ref{fig:pppx}, in the model of power pool, all the structural components of power markets are integrated and coordinated by a single entity that is both market operator and system operator  \cite{Srivastava2011,Barroso2005}, often named independent system operator (ISO). Since scheduling is an integral part of the power market, schedules are determined through a single market gateway, and markets are cleared abiding by the limits of physical deliveries.  In a power pool, ISO seeks to minimize the system total production cost through a centralized unit commitment to fulfill demands economically. Generators must follow the commitment schedule and the dispatch instructions issued by the ISO to receive payments \cite{Kardakos2013}. Otherwise, ISO may charge penalties from the generators or suspend their participation in the power pool. Market activities are mainly on the generation-side, while demands are consolidated as input of ISOs' optimization. Players on the demand-side are usually not able to participate in the market directly unless specific measures are implemented.

In contrast, in the model of power exchange, a transmission system operator (TSO) is still responsible for scheduling coordination, ancillary service provision and transmission system operation, but power transactions are made through a power exchange organized by a third party or through bilateral contracts. Therefore, a market participant is able make electricity transactions in more than one market. As a matter of fact, power exchanges are mostly established by profit-seeking market players and have evolved from the bilateral contract model \cite{Barroso2005}. The system operator usually has no direct control on the power exchange and its role is limited to the physical aspect of maintaining system security. Each producer is responsible for self-scheduling its own units with a decentralized price-based unit commitment \cite{Kardakos2013}. Therefore, power markets organized in power exchange model can be viewed to have a higher level of unbundling than those in power pool model, without invention of physical system operators in electricity trading activities.

Examples of energy markets organized in power pool:
	\begin{itemize}
		\item Most markets organized by ISOs in North America such as PJM, NYISO, Alberta, etc.
		\item Australia's NEM
		\item Ireland's SEM.
	\end{itemize}

Examples of energy markets organized in power exchange
	\begin{itemize}
		\item most markets in European countries such as Germany, Nord Pool, GB etc.
		\item CAISO in the US.
	\end{itemize}

\subsubsection{Implications for flexibility solutions}
In power exchange, the participation from supply-side and demand-side is generally symmetric and offers/bids are usually in the single form of price-quantity pairs. The physical realization of delivery is unbundled  from market activities and is not concerned by market operators. This allows great freedom for flexibility players to participate in the market, regardless of whether the flexibility comes from supply- or demand- side or mixed, and which technologies are employed. 

In power pool, however, generators are usually required to submit complex unit offers including physical information of resources, e.g., unit start-up and shut-down procedures, minimum-up/down time constraints, min/max power output restrictions, ramp-rate limits, transmission limits etc. \cite{Kardakos2013}, and participation from demand-side is generally limited. Therefore, with bundling physical and market activities, participation of flexibility is extensively under control of power pool operators. Being recognized as a generation resource or special market gateway for demand-side participation is necessary prerequisite for a new flexibility resource to directly participate in markets. Otherwise, it would be only limited to behind-the-meter applications where some flexibility resources such energy storage can complement with existing resource or load to adjust a player's position in market. In this way, the operation of flexibility might not be optimal and aggregation is impossible. In addition, due to the strong position of power pool operators, it is less likely for players to gain market power than in power exchange.

Overall, there are greater limits for market participation of flexibility solutions in power pool than in power exchange. Technology vendors need to go further in their efforts aligning market rules and regulatory environment for business planning in power pools. 

\subsection{Marketplace}

In most regions, the energy market consists of several marketplaces along the timeline, as shown by Figure \ref{fig:energy-marketplaces}\footnote{Forward products are excluded since financial derivative markets are out of our scope; refer to Chapter \ref{ch:introduction}.}.

\begin{figure}[h!]
	\centering
	\includegraphics[width=0.95\linewidth]{Figures/EnergyMarketplaces}
	\caption{Typical marketplaces in wholesale energy market}
	\label{fig:energy-marketplaces}
\end{figure}

In markets organized with power exchange model, large volumes of energy are usually traded in day-ahead (DA) market. Intra-day (ID) market, which can be viewed as an extension of day-ahead spot market bringing gate closure near delivery, is a common measure to mitigate increasing needs of real-time balancing operations, as introduced in Chapter \ref{ch:introduction}. All deviations from the commitment scheduled by DA and ID markets requires balancing energy delivery that is coordinated by system operators and are settled through a third marketplace, often named balancing energy market. Imbalance settlements that are accounted in the balancing energy market involve two mechanisms: first, the deviation of one player can be somehow offset by opposite deviations of other players; second, on the system level, the aggregated imbalance is settled by activating frequency control services. In most market regimes, system operators will play a centralized role to clear and settle costs incurred from both mechanisms, while sometimes system operators may allow ex-post trading between market players regarding the imbalance settlement through the first mechanism such as the Swiss and Greek power markets.

Slightly different arrangements are adopted in power pools. Since delivering balancing energy is the responsibility of the  same entity that operates the energy markets, real-time markets are used for settlements of both post-DA scheduling adjustments and balancing operations.

The three-settlement market (i.e. day-ahead, intra-day and balancing market) is the European Union target electricity model \cite{EuropeanCommission2016} so has been implemented in most European energy markets such as Germany, France, Denmark, GB, Italy, Spain, etc.  Two-settlement market (i.e. day-ahead and real-time market), on the other hand, is a common practice in North America  \cite{Cochran2013}.

%\subsubsection{Implications for flexibility management}
Generally, arbitrage in DA market is less favorable for emerging flexibility players, due to relative low volatility and dominance of large conventional generation companies. Flexibility solutions shall gain more advantage in market closer to delivery due to its comparative competence of fast response and operations to conventional generators. 

Nonetheless, participation in any marketplace to perform arbitrage is potentially profit-making. Therefore, identifying which marketplaces exist and whether they are accessible is a necessary step for valuing the opportunities of flexibility solutions.

\subsection{Pricing scheme}
If a marketplace is accessible for flexibility players, a further concern would be the profitability of arbitrage. Since arbitrage is essentially a game played with prices, the pricing mechanism is of most importance, which is however highly diverse across different markets.

\subsubsection{Nodal pricing vs. zonal pricing}
With nodal pricing scheme, prices at each network node are different. On the contrary, uniform pricing scheme applies same price everywhere in the whole control area. Zonal pricing as a trade-off between these two schemes, use the same price in a particular zone including a bundle of nodes.

Nodal pricing internalizes network congestion in price formation. If congestion restricts lowest-cost electricity being transmitted to a particular location, electricity with higher cost but no congestion is dispatched and consequently price at that location will rise. Nodal pricing has clear benefits \cite{Wang2015} but it is harder to implement, especially in markets arranged in power exchange where market operators have no insight into the physical system\footnote{In these regions, it is possible to implement nodal pricing in balancing markets that is coordinated by physical system operators, as illustrated by a research project \cite{Ecogrid}. However, we have not seen any large-scale practice in reality.}. 

Nodal pricing is adopted in many systems in North America, such as PJM, CAISO, NYISO, ISO-NE etc., using a mechanism named locational marginal price (LMP) model. Zonal pricing is used in Australia's NEM and other energy markets organized in power exchange model. 

Nodal pricing incorporates the consideration of congestion. The value of T\&D congestion relief can theoretically be partially captured by arbitrage, especially using flexibility technologies that are easier to be deployed at smaller scale in particular locations such as batteries. However, for aggregators, nodal pricing increases the operational complexity.

\subsubsection{Time resolution}
Since RES generation is intermittent and may vary significantly in a short time interval so may the residual load. As a result, a higher time resolution of pricing can better represent the market need for flexibility. Emerging flexibility solutions with faster response and higher ramp rate shall gain advantages with higher pricing resolution in theory. However, it should be noted that the pricing and dispatching time interval is sometimes different to the settlement interval. For example, the real-time markets in PJM has 5-min pricing resolution but settlement of energy delivery is accounted at hourly resolution \cite{PJM2017}. In such an arrangement, arbitrage against the original price signals may be activated for price differences within a settlement interval, which brings no revenue. Therefore, the operational plan of arbitrage should be determined based on estimation of prices for actual settlement.


\section[Flexibility solutions in ancillary service markets]{Flexibility solutions in ancillary service markets%}
	\sectionmark{Ancillary service market}}
\sectionmark{Ancillary service market}
\label{sec:market-as}
Among all ancillary services, this thesis is particularly focused on frequency control services that are used to tackle imbalance between supply and demand by delivering balancing energy, as introduced in Chapter \ref{ch:introduction}. Frequency control services are usually the most costly among all ancillary services and relying on services provision from market players, while there are usually no markets for other ancillary services such as voltage support, loss compensation, black start etc \cite{Rebours2009,Cochran2013}. 

In different regimes, there exist many differences regarding how frequency control services are defined, procured and operated, as well how the cost is allocated and recovered. Understanding these differences allows technology vendors know which services can be provided using flexibility and to whom they can sell flexibility solutions.

\subsection{Terminology for frequency control services}
Different terminologies used in different power jurisdictions may easily lead to confusion while comparison between different regimes is to be made. Different terms are often used to refer to the same service, while in some instances the similar terms may refer to two disparate services in different regimes. For example, secondary control reserve (SCR) and automatic frequency restoration reserve (aFRR) (both used in European markets) are interchangeable concepts. On the contrary, primary reserve in North America is often used to distinguish services from supplementary reserve, while it is closer to the concept of teirtiary reserve rather than primary reserve used in Europe.

Generally, these terminologies can be classified into two groups as they follow the guidance of service definitions from the Federal Energy Regulatory Commission (FERC) and the Union for the Coordination of the Transmission of Electricity (UCTE). According to the functioning mechanism\footnote{ PCR refers to response activated locally by a speed governor fitted in generator. SCR is activated by a centralized control signal named automatic generation control (AGC) signal. TCR follows manual orders from system operators \cite{EllisonJ.F.TesfatsionL.S.LooseV.W.Byrne2012}. }, terminologies in these two systems can be mapped into a a comparison framework show by Table \ref{tab:term}.

\begin{table}[h!]
	\footnotesize
	\centering
	\begin{tabular}{L{3cm} L{3cm} | L{3cm} L{3cm}}
		\hline
		\hline
		\textbf{UCTE terms} &  \textit{Equivalents} & \textbf{FERC terms} &  \textit{Equivalents} \\
		\hline
		\hline
		Primary control reserve (PCR) & Frequency containment reserve (FCR) & Frequency response & \\
		\hline
		Secondary control reserve (SCR) & Automatic frequency restoration reserve (aFRR) & Frequency regulation & \\
		\hline
		\multirow{3}{3cm}{Tertiary control reserve (TCR)}
		  & \multirow{3}{3cm}{Manual frequency restoration reserve (mFRR)} & Spinning reserve & Synchronous reserve \\
		  \multirow{3}{3cm}{}& \multirow{3}{3cm}{}& Non-spinning reserve & Non-synchronous reserve/ Quick-start reserve \\
		  \multirow{3}{3cm}{}&\multirow{3}{3cm}{} & Supplemental reserve & Replacement reserve \\
		  \hline
		  \hline
	\end{tabular}
\caption{Terminology for frequency control reserves in various regimes \cite{Rebours2009,EllisonJ.F.TesfatsionL.S.LooseV.W.Byrne2012,Wang2015}}\label{tab:term}
\end{table}

It shall be noted that in terms of activation time, UCTE has specifically defined that:

\begin{itemize}
	\item Primary reserve shall be automatically activated within 30s;
	\item Secondary reserve need to be completely delivered within 15 minutes; 
	\item Tertiary reserve shall start within 15-20 minutes after received the order from system operators.
\end{itemize}

In contrast, the time framework for each service category is not aligned among markets in North America \cite{EllisonJ.F.TesfatsionL.S.LooseV.W.Byrne2012}, but generally, activation time of frequency regulation is comparable to an in-between state of primary and secondary control reserve. 

In addition, there are no markets for frequency response in North America \cite{Rebours2009,EllisonJ.F.TesfatsionL.S.LooseV.W.Byrne2012} that are equivalent to primary control reserve markets in Europe.

Generally, new flexibility solutions have advantages for services with shorter activation time and shorter duration compared to service providers using conventional generation. Therefore, frequency control services can be roughly ranked in accordance with the extent to which they are suited to emerging technologies, from most to least: primary, secondary and tertiary. However, this is case-specific depending on characteristics of specific technologies and markets, so is not discussed in details here.

\subsection{Procurement and cost allocation}

Usually, markets for frequency control services involve trading for two commodities, i.e. capacity and energy, as shown by Figure \ref{fig:FCR_market}. 

\begin{figure}[h!]
	\centering
	\includegraphics[width=0.95\linewidth]{Figures/FCR_market}
	\caption{Illustration of markets and activities related to frequency control services}
	\label{fig:FCR_market}
\end{figure}

Capacity refers to a commitment that service providers make to system operators, that they will keep reserves ready to be dispatched for real-time operations. The requirement for capacity is determined by the system operator and procured ahead of real-time operation. Specifically, in the continental European synchronously interconnected system, a total PCR of 3000 MW needs to be provided according to the rules of the European Network of Transmission System Operator (ENTSO-E), while amounts of necessary SCR and TCR capacity are determined by each TSO \cite{ENTSO-e_handbook}. For instance, in Germany, TSOs run a quarterly assessment process to dimension the provision of SCR and TCR for next three months \cite{ConsentecGmbH2014}. In North America, ISOs determine the need for reserve capacity by conducting their own processes, so-called reliability assessment, which take place after gate closure of day-ahead market and before each operating hour\cite{EllisonJ.F.TesfatsionL.S.LooseV.W.Byrne2012}. 

Energy is what services providers actually deliver to the system upon activation by system operators in real time. The amount of energy is determined based on physical needs for grid balancing.

The acquisition and settlement process for the frequency control capacity and energy also varies amongst different market regimes.

Generally, two models are identified. We name them as centralized procurement and decentralized procurement respectively; see Figure \ref{fig:FCR_market-model}.

\begin{figure}[h!]
	\centering
	\includegraphics[width=1.05\linewidth]{Figures/FCR_market_model}
	\caption{Two models for procurement and cost allocation of frequency control services}
	\label{fig:FCR_market-model}
\end{figure}

In the centralized model, the system operator is designated as the single buyer \cite{Rebours2009}. System operators (SO) will either organize auctions in short term ahead of the operating day, e.g. German TSOs organize weekly-ahead auctions for SCR and day-ahead auctions for TCR \cite{ConsentecGmbH2014}, or seek long-term bilateral contract with service providers, e.g. Australian Energy Market Operator (AEMO) uses this approach to organize ancillary services in NEM \cite{AEMO2015}. On the other hand, SOs need to recover costs incurred by charging entities with obligation. In different markets and for disparate services, obligations are assigned in various ways. For example, costs for energy of frequency control services in Germany and for regulation reserve in Australia are recovered from entities who violate their commitments determined in energy markets, while costs for capacity of frequency control services in Germany and for contingency reserve (similar to PCR) in Australia are socialized among all market participants. 

Decentralized model is adopted by ISOs in North America. In this model, ISOs allocate requirements for reserve capacity to market players according to their servicing loads to the system total load \cite{Rebours2009,EllisonJ.F.TesfatsionL.S.LooseV.W.Byrne2012,PJM2017b}. Market players have to fulfill their own obligations through self-supplied reserve, through bilateral contract with other market participants, and/or through purchases of reserve in some form of reserve market organized by ISO \cite{EllisonJ.F.TesfatsionL.S.LooseV.W.Byrne2012}. In this way, market participants in the power market are put into competition for procuring frequency control services. Examples using this model include all seven ISOs in the US.

Flexibility solutions can be employed for the provision of frequency control services and for fulfilling obligations in both arrangements. However, while provision and obligation fulfillment are symmetric in the decentralized approach, there might be asymmetry in the centralized approach with SOs standing in-between. Payments may differ between services provided for SOs and for market players to fulfill their obligations.

\subsection{Frequency control product design}
Further to the high-level distinctions mentioned previously, attentions should also be paid to some key details regarding how the frequency control service as products are designed. Product design will significantly affect the feasibility and profitability of certain technologies providing frequency control services. Without mentioning too many technological specifications, we discuss four points here. 

First of all, pre-qualification of resources to provide a given service is necessary. While activation time is usually an advantage of emerging flexibility solutions, duration of dispatch tends to be a bottleneck, especially for tertiary control reserves. For instance, CAISO requires a minimum of 30 minutes duration for delivering spinning and non-spinning reserves and duration of providing tertiary reserve for German TSOs is in 6-hour blocks. In these cases, some flexibility solutions, such as flywheel energy storage that is only able to last for about 15 minutes \cite{EllisonJ.F.TesfatsionL.S.LooseV.W.Byrne2012,Beaudin2014}, are excluded from provision of those services.

Second, frequency control services are sometimes divided into up and down services. Up services mean there are generation shortage and injection of energy or reduction of demand are required. On the contrary, down services refer to situations where more demand or less generation is needed. Separate markets for these two types of services would allow more choices for flexibility players to make optimal offers in accordance of the technological characteristics of their flexibility resources.

Besides, it is of a concern how automatic frequency control signals are engineered. For example, an energy storage device that does not generate energy will favor a signal that is energy-neutral to it, i.e. the state of charge of the device can come back to its initial value after a period of operation.

Finally, one should consider how services are priced. Capacity commitment and actual delivery, i.e. amount of released energy and sometimes performance as well, are normally priced and settled separately. Since flexibility solutions have the potential to outperform conventional flexibility solutions considering their technological characteristics of fast response and high ramp rate, a pricing scheme where performance of delivery is valued tends to offer merits for emerging flexibility solutions. By this rationale, the FERC requires ISO markets to compensate for regulation based on actual service provided
according to its Order 755 \cite{FERC755}. Some ISOs including PJM, NYISO, ISO-NE followed the order to establish such a mechanism. Nevertheless, in most market regimes, only amount of energy is accounted for final payment for frequency control services.

More detailed impact of product design and  technical implications will be discussed in Chapter \ref{ch:cases}.

\section[Flexibility solutions in capacity markets]{Flexibility solutions in capacity markets%}
	\sectionmark{Capacity Market}}
\sectionmark{Capacity Market}
\label{sec:CM}

The capacity market is established in some power market jurisdictions to minimize investment risks of power generators so that resource adequacy can be effectively ensured. Investors are remunerated for commitment to keep capacity online. However, it is not a common practice, because of complex political reasons which are not our focus in this thesis, but it is worth to mention briefly that ensuring minimal investment risk for generators means risks are somehow shifted to consumers \cite{Cochran2013}. 

However, for flexibility players and technology vendors, the existence of capacity market is generally favorable as it potentially provides a direct revenue stream. Naturally, one should examine which technologies are suitable and whether demand-side resources are qualified to receive remuneration.

Examples of power market jurisdictions with capacity market include PJM, NYISO, ISO-NE, Spain, Ireland, GB (since 2014), etc. Also, transition from an energy-only market towards a capacity market has been observed in some markets, e.g. Ontario IESO and Alberta AESO are in the process of developing a capacity market, initiated in 2014 and 2016 respectively. 

In energy-only markets, system operators have sometimes taken alternative measures to ensure adequacy, e.g. strategic reserves or named emergency products. Strategic reserves and emergency products are either generation capacity or curtailable loads that are activated only when scarcity of generation is observed (typically reflected by extremely high prices).

Energy-only markets with such capacity remuneration mechanisms include: ERCOT, Australia's NEM, Germany, Nord Pool, Belgium etc. 

Finally, for markets without any of these capacity measures mentioned above, it is likely for extreme prices to occur, which will be an indirect incentive for flexibility players' arbitrage in energy markets.

\section{Aggregator and demand-side participation}

Participation of aggregators and other providers of demand-side flexibility is not always allowed in some marketplaces. This is especially an issue for ancillary services and capacity markets as they were initially designated only for generation resources. Participation in energy markets may also be limited in power pool arrangement as discussed previously. Therefore, it is of great importance to examine the market rules regarding this issue.

So far, examples of power pools that allow participation of aggregators and demand-side responses in energy market include: PJM, ISO-NE, Ontario IESO, Singapore and Australia's NEM (as of April 2017 \cite{AEMO_DR}), etc.

Examples of jurisdictions allowing participation of aggregators and demand-side responses in frequency control ancillary service market include: PJM, Ontario, Singapore, Alberta AESO, ERCOT, Australia's NEM, etc.

Examples of capacity markets that have remuneration programs for demand-side resources include: PJM, ISO-NE.

Examples of energy-only market that have strategic reserve of emergency products for demand-side resources include: ERCOT, Australia's NEM, Nord Pool, Germany, etc.

\section{Summary and the analytical framework}

From the analysis presented above, it is clearly seen that investigating opportunities for flexibility solutions across different market regimes is indeed a sophisticated task since many layers of hierarchy exist in terms of structural differences across markets. In order to better guide technology vendors for qualitative assessment of flexibility solution in a given regime, we organized the previous analysis into analytical frameworks illustrated by Figure \ref{fig:qualitative-energy}-\ref{fig:qualitative-capacity}. We apply these frameworks for our own analysis in the reminder of this thesis.

\begin{figure}[h!]
	\centering
	\includegraphics[width=0.95\linewidth]{Figures/Q_energy}
	\caption{Analytical framework for qualitative analysis of flexibility solutions in energy market}
	\label{fig:qualitative-energy}
\end{figure}

\begin{figure}[h!]
	\centering
	\includegraphics[width=0.95\linewidth]{Figures/Q_frequency_control}
	\caption{Analytical framework for qualitative analysis of flexibility solutions in frequency control market}
	\label{fig:qualitative-fr}
\end{figure}

\begin{figure}[h!]
	\centering
	\includegraphics[width=0.95\linewidth]{Figures/Q_capacity}
	\caption{Analytical framework for qualitative analysis of flexibility solutions in capacity market}
	\label{fig:qualitative-capacity}
\end{figure}


\chapter{Methodology for Quantitative Valuation of Flexibility Management Markets}
\label{ch:methodology}
\textit{This chapter presents the methodology for quantifying the value of flexibility managment markets. A modular approach is adopted to overcome the complexity from multi-dimensional market-technology contexts. Firstly, the modules are introduced, being categorized into market- and technology- based groups. Then we will explain how these modules are to be organized within a optimization.}

\section{Modular approach to build valuation models}
In this thesis, a list of different markets and two different technologies are being studied. This results in a significate number of cases of environment. It is not possible to generalize the model for these cases due to multi-dimensional structural differences. On the other hand, building a model for each case will lead to redudancy and make the model less usable and harder to maintain. Therefore, we adopt a modular approach where the dynamics of markets (or technologies) are generalized and varable in market-based (or technology-based) modules. The modular approach does not reduce the complexity of the problem, but renders the model more structurally organized.

\begin{table}
	\label{tb:modules}
	\begin{center}
		\begin{tabular}{|L{1.5cm}| L{2.75 cm} | L{4 cm} | L{4cm} |}
			\hline
			\textbf{Section} &\textbf{Module name} & \textbf{Input} & \textbf{Output} %& \textbf{Parameters}
			\\[0.5ex]
			\hline \hline
			\multicolumn{4}{|c|}{Market-based modules }\\
			\hline
			4.2.1&Revenue module & Price signals (Determinate part), Frequency control singals, Sets of targeted marketplaces & Matrix of coefficients for revenue calculation %& None 
			\\
			\hline
			4.2.2&Risk module & Price signals (Distribution of stochastic part), Frequency control singals, Sets of targeted marketplaces& Matrix of coefficients for calucating Conditional Value-at-Risk %& Confidence level 
			\\
			\hline
			4.2.3&Market simulation module & Generation by fuel type, consumption and its elasticity& Price and volume signals% & The parameters were  obtained by regression with historical data 
			\\
			\hline
			4.2.4&Market constraints & Volume signals & Constraints for optimization %& Market rules 
			\\
			\hline
			\hline
			\multicolumn{4}{|c|}{Technology-based modules }\\
			\hline
			4.3.1&Cost module & Investment cost, Designed life time, Operating life time, System state & Matrix of coefficients for cost calculation %& None
			\\
			\hline
			4.3.2&Technology simulation module & Efficiencies of charging, discharging and storing; Capacity; Energy-to-power ration& Matrix of coefficients to determine system states %& 3
			\\
			\hline
			4.3.3&Technology constraints & Historical data (Generation by fule type, consumption, market price and volume)& Price and volume signals %& 3 
			\\
			\hline
		\end{tabular}
	\end{center}
	\caption{List of modules}
\end{table}

\begin{figure}[h!]
	\label{fig:model-flow}
	\includegraphics[scale=0.4]{Figures/ModelFlow.pdf}
	\caption{Flow chart of the techno-economic model}\
\end{figure}

Table \ref{tb:modules} offers an overview of all the modules and their inputs and outputs. The working flow of the model is illustrated by Figure \ref{fig:model-flow}.

With this model, we can evaluate the profitability and risk associated with a certain scale of flexibility management system in the power market and thus estimate the value of flexibility management market. Furthermore, we can assess the impact of driving factors including renewable penetration, cost reduction, and the possible diminishing return with increasing flexibility.

\section{Market-based modules}

\subsection{Revenue module}
\label{sec:revenue}
In this study, we only consider explicit revenues from power markets. At each time step ($t$), the revenue ($\text{REV}_t$) is calculate as the amount of energy ($e_t$, in MWh) offered in each energy market segment ($i$), and/or amount of reserve ($r_t$, in MW) offered in each reserve market segment ($j$), multiplied by their corresponding prices ($\pi_t$, in \$/MWh or \$/MW). In reserve market, there are additional revenues from energy provision while the committed capacities are activated. The amount of energy delivered in reserve market is determined as a proportion of the committed reserve using a term of ratio ($\delta_t$, in MWh/MW). The total revenue within a given period of time ($T$) and a set of selected energy markets ($I$) and a set of  selected reserve markets ($J$), can be then computed as:

\begin{equation}
\label{eq:module-revenue}
\text{REV}=  \sum_{t}^{t \in T} \text{REV}_t = \sum_{t}^{t \in T} \left( \sum_{i}^{i \in I}  \pi_t^{e,i} (e_t^{d,i} - e_t^{c,i})  + \sum_{j}^{j \in J} (\pi_t^{e,j} \delta_t^{j} + \pi_t^{r,j}) r_t^j \right)
\end{equation}

where, $d$ and $c$ in the superscripts denote "discharge" (to release energy from flexibility resources to grids) and "charge" (to intake energy from girds to flexibility resources) respectively.  $e_t^{d,i}$, $e_t^{c,i}$, $r_t^{j}$, are endogenous variables of the whole model and decision variables of the optimization, which represent the operation plan of the flexibility resource in power markets.

$I$ and $J$ are determined according to the business case being studied. For example, we can set $I = \{Day~ahead\}$ and $J=\emptyset$ in order to the value of making arbitrage in day-ahead energy market. 

If there are multiple elements in $I \cup J$, it means the flexibility resource can be reallocated to make offers to different market segments, i.e. performing multitasking. These cases need to be carefully managed to comply with actual market rules. Detailed treatments regarding multitasking are illustrated in section \ref{sec:special}.

The ratios $\delta_t$ are computed based on the real control signal when data is available, or otherwise using system average ratios between total activated energy ($\hat{e}_t^{r,j}$) and the total reserve ($\hat{e}_t^{r,j}$) at each time step.

%\begin{equation*}
%\delta_t^j = \frac{\hat{e}_t^{r,j}}{\hat{r}_t^j}
%\end{equation*}

Price signals, $\pi_t^{e,i}$, $\pi_t^{r,j}$ and $\pi_t^{e,j}$, are inputs for the revenue module and may be retrieved either directly from historical data or from the outputs of market simulation module described in Section \ref{sec:market-simulation}.

We re-formulate Equation \eqref{eq:module-revenue} in form as:

\begin{equation*}
\text{REV} = \textit{\textbf{f}}~X
\end{equation*}

where $X$ is the vector for all desicion variables. For certain sets of market segments $I$ and $J$, $X$ can be derived using Equations \eqref{eq:decision-variable-1} - \eqref{eq:decision-variable-end} with $i \in I$ and $j \in J$.
\begin{equation}
\label{eq:decision-variable-1}
X =
\begin{bmatrix}
E^d \\ E^c \\ R
\end{bmatrix}
\end{equation}

\begin{equation}
E^d =
\begin{bmatrix}
E^{d,I(1)} \\ \vdots \\ E^{d,i} \\ \vdots \\ E^{d,I(|I|)}
\end{bmatrix} \\
E^{d,i} = 
\begin{bmatrix}
e_1^{d,i}\\e_2^{d,i}\\\vdots\\e_T^{d,i}
\end{bmatrix}
\end{equation}

\begin{equation}
E^c =
\begin{bmatrix}
E^{c,I(1)} \\ \vdots \\ E^{c,i} \\ \vdots\\ E^{c,I(|I|)}
\end{bmatrix} \\
E^{c,i} = 
\begin{bmatrix}
e_1^{c,i}\\e_2^{c,i}\\\vdots\\e_T^{c,i}
\end{bmatrix}
\end{equation}

\begin{equation}
\label{eq:decision-variable-end}
R =
\begin{bmatrix}
R^{J(1)} \\ \vdots \\R^{j} \\ \vdots \\ R^{J(|J|)}
\end{bmatrix} ~~~\\
R^{j} = 
\begin{bmatrix}
r_1^{j}\\r_2^{j}\\\vdots\\r_T^{j}
\end{bmatrix}
\end{equation}

Function $\textbf{f}$ can be obtained analogously using Eqution \eqref{eq:decision-f-revenue-1} $\sim$ \eqref{eq:decision-f-revenue-end} with $i \in I$ and $j \in J$.
\begin{equation}
\label{eq:decision-f-revenue-1}
\textit{\textbf{f}} =
\begin{bmatrix}
\Pi^{e,I}~|~&-\Pi^{e,I}~|~&\Pi^{e,J} \Delta^J + \Pi^{r,J}
\end{bmatrix}
\end{equation}

\begin{equation}
\Pi^{e, I} =
\begin{bmatrix}
\Pi^{e,I(1)}~|~&\dots~|~&\Pi^{e,I(|I|)}
\end{bmatrix} ~~~\\
\Pi^{e,i} = 
\begin{bmatrix}
\pi_1^{e,i}~\pi_2^{e,i}~\dots~\pi_T^{e,i}
\end{bmatrix}
\end{equation}

\begin{equation}
\Pi^{e,J} =
\begin{bmatrix}
\Pi^{e,J(1)}~|~&\dots~|~&\Pi^{e,J(|J|)}
\end{bmatrix} ~~~\\
\Pi^{e,j} = 
\begin{bmatrix}
\pi_1^{e,j}~\pi_2^{e,j}~\dots~\pi_T^{e,j}
\end{bmatrix}
\end{equation}

\begin{equation}
\Pi^{r,J} =
\begin{bmatrix}
\Pi^{r,J(1)}~|~&\dots~|~&\Pi^{r,J(|J|)}
\end{bmatrix} ~~~\\
\Pi^{r,j} = 
\begin{bmatrix}
\pi_1^{r,j}~\pi_2^{r,j}~\dots~\pi_T^{r,j}
\end{bmatrix}
\end{equation}

\begin{equation}
\label{eq:decision-f-revenue-end}
\Delta^J = diag (
\delta_1^{J(1)}, \dots , \delta_T^{J(1)}, \dots, \delta_1^{J(|J|)}, \dots, \delta_T^{J(|J|)})
\end{equation}

%\subsection{Risk module}
In accordance with the revenue calculation, we consider the uncertain movement of price as the primary source of risk. Referring to similar works that performed risk management for flexibility sources, e.g. EV2G \cite{Alipour2017} and DER \cite{Han2017}, as well as for conventional energy trading companies \cite{Mohammadi-Ivatloo2013}, we developed a simple measure for risk control, by using the conditional value-at-risk (CVaR).

The CVaR (also named expected shortfall) as an extension of value-at-risk (VaR) can be defined as the difference between the expected profit and the average of potential profit values which are less than VaR \cite{Rockafellar2000}, shown as:

\begin{equation}
\label{eq:CVaR}
CVaR_\alpha (X) = \int_{\alpha}^{1} VaR_s(X) ds
\end{equation}

where $\alpha$ is the confidence level, and $X$ is the underlying (the price of energy/ reserve in our study). The VaR, as the negative of $\alpha$-quantile, can be computed as:

\begin{equation}
\label{VaR}
VaR_\alpha(X) = inf \{x \in \mathbb{R}~|~ P(X+x<0)\leq 1-\alpha\}
\end{equation}

Specially, in case the underlying variable subject to normal distribution, i.e. $X \sim \mathcal{N}(\mu,\,\sigma^{2})\,$, we can derive the CVaR as:

\begin{equation}
CVaR_\alpha(X) = \mu - \sigma \frac{\phi(\Phi^{-1}(\alpha))}{1-\alpha}
\end{equation}
%http://blog.smaga.ch/expected-shortfall-closed-form-for-normal-distribution/
where, $\Phi(\cdot)$ is cumulative distribution function and $\phi(\cdot)$ is the probability density function of normal distribution.

Alternatively, if the uncertainties are dealt with in a discrete manner, the CVaR can be calculated as\cite{Rockafellar2000}:

\begin{equation}
CVaR_\alpha (X) = \underset{\zeta}{max}\left( \zeta - \frac{1}{1-\alpha} \sum_{s} P(X,s) (\zeta - f(X,s))\right)
\end{equation}
where, $P(X,s)$ is the probability distribution function of $X$ in the scenario $s$ and $f(X,s)$ is the profit function in the scenario $s$. $\zeta$ is an auxiliary variable constrained by

\begin{equation*}
\zeta - f(X,s) \leq \zeta_s
\end{equation*}
\begin{equation*}
\zeta_s \geq 0
\end{equation*}

In our study, price terms $\tilde{\pi}$ are assumed to comprise a determinate part $\pi$ and an independent stochastic deviation $\epsilon$:
\begin{equation}
\label{eq:price-error}
\tilde{\pi_t}= \pi_t + \epsilon_t
\end{equation}

Since the stochastic terms $\epsilon$ are assumed to be uncorrelated to each other, the CVaR of our portfolio that is built by  $X^T = [E^d~|~E^c~|~R]$
in Equation \eqref{eq:decision-variable-1} can be aggregated as:

\begin{equation}
\begin{aligned}
CVaR =\sum_{t}^{t \in T} \{&\\
&\sum_{i}^{i \in I}  CVaR(\tilde{\pi_t}^{e,i}) (e_t^{d,i} - e_t^{c,i})  \\
&+ \sum_{j}^{j \in J} \left(CVaR(\tilde{\pi_t}^{e,j}) \delta_t^{j} + CVaR(\tilde{\pi_t}^{r,j})\right) r_t^j \\
&\}
\end{aligned}
\end{equation}

Analogous to the formation in preceding section, the risk module is also formulated in vector and matrix form.

\begin{equation*}
CVaR= \textit{\textbf{f}}
\begin{bmatrix}
E^d \\ E^c \\ R
\end{bmatrix}
\end{equation*}

where \textit{\textbf{f}} is calculated as:

\begin{equation}
\textit{\textbf{f}} =
\begin{bmatrix}
CVaR(\Pi^{e,I})\\-CVaR(\Pi^{e,I})\\CVaR(\Pi^{e,J}) \Delta^J + CVaR(\Pi^{r,J})
\end{bmatrix}^T
\end{equation}

\subsection{Market simulation module}
\label{sec:market-simulation}
As has been illustrated in the literature review (Chapter \ref{ch:LitRev}), valuation of flexibility with a dynamic market condition is still a challenging task. While investment decisions are extensively concerned with long-term trends, profitability of arbitrage sensitively depends on short-term price movement in high resolution. This is distinguishing from conventional electricity generators for whom a long-term forecast with coarse resolution is sufficient, and visual arbitrageurs who have almost no investments on infrastructures and may perform decision-makings with a short-term perspective. A holistic approach combining these researches were taken sometimes \cite{Rastler2010}\cite{Eyer2010} but may easily bring in unnecessary complexity and lead to an overwhelming demand of resources, which are not essential for our study.

Therefore, in this thesis, we customized a market model based on existing researches by re-focusing on factors that are most relevant to our research questions, and simplifying many other aspects of the power system and markets. Our market model is generally a statistic model built on observations of historical data, but a physical sub-model is incorporated as well to study the impacts of some relevant variables whose features are not well captured by empirical observations.

The approach for market simulation differentiates between energy markets and reserve markets. 

The energy markets are usually matured and with abundant degree of competition, so that we can employ an idealistic market model where the price formation is governed by the short run marginal costs (SRMCs) \cite{Grunewald2012} \cite{Grunewald2012a}. This allows us to leverage a merit-order model to simulate the price levels, which are widely adopted as is summarized in Chapter \ref{ch:LitRev}. 

The design of reserve markets, on the contrary, is not as straightforward as energy markets, which pose challenges for robust modeling. Besides, the market mechanisms vary spatially and temporally as is analyzed previously. Therefore, we adopt a pure statistic model for reserve market without involving any physical modeling.

\subsubsection{Day-ahead energy market}

The simulation for day-ahead energy market is preliminarily based on work done by \cite{Grunewald2012a} where the merit-order curve at supply shortage and surplus is modeled by an uplift effect. We further extend this work to capture the limits of flexibility provision in current energy markets so that we can simulate the market conditions when the flexibility become a challenge with growing renewables and/ or the flexibility becomes ubiquitous.

In \cite{Grunewald2012a}, the peak price during periods of high demand is explained as fewer participants remain with spare generating capacity, putting these actors in a stronger bidding position to mark up the price. In contrast, when demand is low and plants with high SRMCs would not operate so further reduction in generation would favor plants with low SRMCs and thus reverse the bidding position. In both cases, the less available capacity remains, the stronger bidding position for the remaining players, which happens at the two end of merit-order curve where the prices are driven up or down to significantly depart from the marginal cost. The symmetric effect is model with a uplift function:

\begin{equation}
U_t^g = 1 + \kappa~e^{-\alpha\left(\frac{C_t^g -P^g_t }{C^g}\right)}
\end{equation}

where $g$ denote the class of generation in merit order, e.g. peak, flexible, inflexible, etc. ($\kappa$) and ($\alpha$) are the parameters which can be obtained empirically \cite{Cox2009}. In case of peak period, $C^g_t$ represents total avaiable generation capacity of class $g$ and $P^g_t$ is the output of genertion of class g. During period of generation surplus, $C^g_t$ is the remaining generation capacity while $P^g_t$ is the curtailment required.

The middle of merit order curve can be modeled with a linear relationship.

Since the SRMCs of renewable generations are almost zero or even negative when they are remunerated by renewable support schemes, their position in power market is distinguishing from other generation players. Therefore, we employed the residual load, i.e. the load net of renewable generation, which has been introduced previously. We denote the residual load as $L^{res.}$ here.

According to the discussion above, the uplifts will occur when $L^{res.}$ exceeds the capacity of mid-merit generations and when $L^{res.}$ is smaller than operating capacity of inflexible generations.

Therefore, the merit order model for price formation can be formulated as:

\begin{equation}
\label{eq:merit-order-model}
\pi_t = \begin{cases}
\dot{\pi}_t \left[1 + \kappa~e^{-\alpha\left(\frac{C_t^g -P^g_t }{C^g}\right)} \right] & L^{res.}_t \leq C^{inflex.}_t\\ 
\dot{\pi}_t ~\kappa~\frac{P^g_t}{C_t^g} & C^{inflex.}_t < L^{res.}_t < C^{inflex.}_t+ C^{mid.}_t\\
\dot{\pi}_t  \left[1 + \kappa~e^{-\alpha\left(\frac{C_t^g -P^g_t }{C^g}\right)} \right] & L^{res.}_t \geq C^{inflex.}_t + C^{mid.}_t\\
\end{cases}
\end{equation}

In order to derive the value of generation capacity of each class, an investigation into the flexibility of power plants is necessary.

\begin{figure}[h!]
	\label{fig:power-plant-flexibility}
	\includegraphics[scale=0.4]{Figures/PowerPlantFlexibility.pdf}
	\caption{Qualitative representation of key flexibility parameters of a power plant\cite{AgoraEnergiewende2017}}\
\end{figure}

The flexibility of a power plant can be characterized by three key features\cite{AgoraEnergiewende2017} (Figure \ref{fig:power-plant-flexibility}): 
\begin{itemize}
	\item Overall bandwidth of operation: the range of output between minimum and maximum load;
	\item Ramp rate: the speed of adjusting output;
	\item Start-up time: the time required to attain stable operation from standstill
\end{itemize}

%https://www.researchgate.net/profile/Alejandro_Hoese
If a power plant can adjust its load from zero to nominate capacity within a time block in the day-ahead market (typically 1 hour), it can be deemed with infinite flexibility in the day-ahead market. This applies to most type generations including solar, wind, hydro and electrochemical systems, etc., except for generations using steam turbines \cite{AgoraEnergiewende2017}, including nuclear, coal-, oil and gas-steam, etc. The gas turbines can be ramped up to full capacity within typically 30 minutes\cite{Siemens}\cite{GE} so can be considered as flexible generation.

For a steam-turbine power plant, the minimum operational load is about 25-60\% of its nominal capacity while the time required to start from standstill is longer than 2 hours \cite{AgoraEnergiewende2017}. Therefore, they are treated as limited flexible sources.

For limited flexible generations, an empirical analysis is performed to determine its bounded flexibility. The procedure for a certain generation source is decribed as following and shown as Figure \ref{fig:bounded-flexibility}:

\begin{figure}[h!]
	\label{fig:bounded-flexibility}
	\includegraphics[scale=0.4]{Figures/BoundedFlexibility.pdf}
	\caption{Schematic illustration of determining bounded flexibility for limited flexible generations}\
\end{figure}


\begin{enumerate}
	\item Make the duration curve of the generation data, and obtain $\overline{c}^{mid.}$ which is the range that the generation source is operating for over 10-99\% of the overall period and $\overline{c}^{inflex.}$ which is the range that the generation is operating of more than 99\% of the time.
	\item Determine the envelop lines which limit the production at time $t+1$ based on production at time $t$. With a certain production $p_{t}$, $p_{t+1}$ is bounded within $\tilde{c}^{flex}$, and there is a range of production $\tilde{c}^{inflex}$ that is not economically viable to be curtailed.
	\item Finally, we find the relationship that map the production at time $t$ to flexible capacity at time $t+1$ as: 
	\begin{equation}
	\begin{aligned}
	c^{inflex.}_{t+1} &= \mathcal{C}^{inflex}(p_t)\\ &=max\{\tilde{c}^{inflex.}_t,~\overline{c}^{inflex.}\}
	\end{aligned}
	\end{equation}
	\begin{equation}
	\begin{aligned}
	c^{flex.}_{t+1} &= \mathcal{C}^{flex.}(P_t) \\ & =min\{\tilde{c}^{flex.}_t+\tilde{c}^{inflex.}_t,~\overline{c}^{mid.}+\overline{c}^{flex}_t \} - \tilde{c}^{inflex.}_t
	\end{aligned}
	\end{equation}
	\begin{equation}
	\begin{aligned}
	c^{peak}_{t+1} &= \mathcal{C}^{peak}(P_t) \\ & =max\{\tilde{c}^{flex.}_t+\tilde{c}^{inflex.}_t -(\overline{c}^{mid.}+\overline{c}^{flex}_t),0 \}
	\end{aligned}
	\end{equation}
\end{enumerate}

When the load exceeds the flexible range of these sources, they are no long able to participate in the bidding so these portion of capacity shall be deducted from the overall capacity for the calculation using Equation \eqref{eq:merit-order-model}.

Finally, a regression is performed to determine the parameters in Equation \eqref{eq:merit-order-model} using empirical observations. The errors between a regressed value $\pi_t$ and an actual value $\tilde{\pi}_t$ would be analyzed as the uncertainty of price movement and used for risk controlling as is discussed in risk module. 

With a established merit-order model for day-ahead energy market, we can re-simulate the price with changed market condition, e.g. altered generation capacity mix.

\subsubsection{Real-time energy market and reserve market}

In electricity markets, large portion of energy is usually traded in day-ahead market \cite{Kardakos2013}. There are significate dependences of the real-time (intraday, balancing) energy price on day-ahead price \cite{Woo2016}. Therefore, for real-time energy prices, we adopt a simplex empirical analysis based on comparing the results from day-ahead price simulation and actual market data:

\begin{equation}
\pi^{RT}_t = \kappa (\pi^{DA}_t + \alpha) + \epsilon_t
\end{equation}
where, $\kappa$ and $\alpha$ are terms to adjust the determinate bias between day-ahead and real-time price, while $\epsilon_t$ represents the stochastic movement of real-time price. 

For reserve market, only an empirical model is used as is discussed previously.

\subsection{Market constraints}

The market constraints are a list of limits to make sure that the operation of flexibility resource (determined by $X$ in Equation \eqref{eq:decision-variable-1}) would not violate the actual market rules and market conditions.

Generally, these constraints can be formulated as

\begin{equation}
\begin{bmatrix}
\Gamma^d~~|& \Gamma^c~~|&\Gamma^r
\end{bmatrix}
X \leq \textit{\textbf{b}}
\end{equation}

Most of the market constraints are derived from the market rules so will be introduced in case studies where specific markets are being studies. 

%Capacity constraints:
%\begin{equation}
%r_t^j \leq \hat{r}_t^j ~~~ j \in J
%\end{equation}

%Since in this study high penetration of flexibility resources will be investigated, 

%Day-ahead
%\begin{equation}
%\hat{e}_t^i - \hat{e}_t^{peak} \leq e_t^{d,i} - e_t^{c,i} \leq \hat{e}_t^i - \hat{e}_t^{base} ~~~ i \in \{Day~Ahead\}
%\end{equation}

%Real-time
%PJM revised order +/- 10% of the DA offer
%Germany volume limited





\section{Technology-based modules}

\subsection{Cost module}
\label{sec:cost}
In this thesis, we categorize all costs into two groups: operation-independent and operation-dependent costs.

\subsubsection{Operation-independent costs}
The first group mainly including the initial capital outlay for purchasing the devices and systems, plus the fixed operating and maintenance (O\&M) costs which include miscellaneous items such as the insurance, employee salaries, etc. 

The initial capital cost for a storage system can be divided into two components: an energy-based component, approximately linear to the energy capacity of the system (denoted $\overline{s}$, in MWh), and a power-based component, approximately linear to the power rate of the system (denoted $\overline{r}$, in MW) \cite{Megel2017}. Additionally, we add a component representing the size-invariant costs such as the cost for software. Thereby, the initial capital cost can be computed as:

\begin{equation}
\label{eq:initial-cost}
C^{ini} = C^s \overline{s} + C^r \overline{r} + C^0
\end{equation}

where, the coefficients can be obtained empirically either by screening actual market data or from literature. In addition, since the system cost for battery storage is falling rapidly, a learning rate of \textit{ca.} 14\% per annum can be taken to build future scenarios\cite{Nykvist2015}.

The initial capital cost is then annualized by using the concept of equivalent annual cost (EAC):

\begin{equation}
C^{EAC} = \frac{C^{ini}}{\frac{1 - \frac{1}{(1+r)^a}}{r}}
\end{equation}

where $r$ is the discount rate and $a$ is the lifespan of the system in number of years.

The discount rate can be established from the Weighted Average Cost of Capital (WACC) which depends on the financial conditions of different players. A typical WACC in the United States is \textit{ca.} 4-6\% for a municipal utility, 7-8\% for a regulated utility and over 10\% for independent power producer\cite{Rastler2010}. In this study, a discount rate of 10\% is taken unless otherwise stated.

For fixed O\&M costs, $C^{fO\&M}$ which is difficult to calculate precisely, an assumption of 2\% of the initial capital cost is taken, referring to \cite{Rastler2010}. The fixed O\&M costs are added directly to the annualized capital cost to get the total fix costs (in \$/year):

\begin{equation}
C^{fix} = C^{EAC} +  C^{fO\&M}
\end{equation}

The annualized fix cost will finally be compared with the operating revenue calculated from other module to assess the profitability.

\subsubsection{Operation-dependent costs}

Operation-dependent costs primarily refer to the degradation costs, which is specially an issue for battery-based energy storage systems\cite{Barre2013}.

However, as has been reviewed and analyzed in \cite{Megel2017}, there exists no single degradation model that is widely accepted among the literature and applicable for all cases, due to the complexity of this problem. The reasons can be summarized as following:
\begin{itemize}
	\item Modelling battery degradation itself is a complex engineering problem as it is affected by a list of physical parameters, including the degree-of-discharge (DoD), state-of-charge (SoC), charging/discharging rate, temperature, etc.\cite{Barre2013}
	\item The choice of degradation model affects the convex relaxation when degradation effects are included in an optimization problem, the model selection is driven by the requirements of mathematical realization. \cite{Megel2017}
\end{itemize}

Degradation costs can be neglected while operating life time is longer than designed life time, which is generally valid for non-battery energy systems \cite{Bradbury2014}\cite{Zafirakis2016}\cite{Connolly2011}. Some research works studying battery system also made the same assumption  \cite{Byrne2012}\cite{McConnell2015}\cite{Sioshansi2009}. The breakeven point of operational frequency where the degradation of battery storage system can be ignored was concluded to be less than 0.5-1.5 full-cycle equivalent energy throughput per day\cite{Megel2017}. Nonetheless, it was also pointed out by \cite{Megel2017} that while assuming degradation cost being zero, the operational planner would tend to operate the system more frequently, which would possibly in turn to violate the assumption of zero-degradation.

Such a combined investment and operation problem is hard to be incorporated in an optimization, so in our study we first use a simple degradation cost model where the cost is linear to the \textit{energy throughput} $|e^t|$ as a damping term in the optimization and examine it \textit{ex-post}, i.e. if the actual operating life is not reached the degradation cost will be exempted from the final profit calculation. A linear relationship between the degradation and $|e^t|$ is a common technique used in researches for estimating battery degradation\cite{Byrne2012}\cite{Berrada2016}.

Denoting the damping factor for degradation as $\zeta$, we can formulate the degradation damping as:

% THIS IS WRONG
\begin{equation}
\label{eq:degradation-damping}
C^{degradation}_t = \zeta (\sum_{i}^{i \in I}(e_t^{d,i}+e_t^{c,i})+\sum_{j}^{j \in J}(\delta_t^{j,+}+\delta_t^{j,-})r_t^j)
\end{equation}

where, the energy to reserve ratios are separated to positive and negative components:
\begin{equation}
\label{eq:ratio-pos}
\delta_t^{j,+} = \begin{cases}
\delta_t^j & \delta_t^j  \geq 0\\
0 & \delta_t^j  < 0
\end{cases}
\end{equation}
\begin{equation}
\label{eq:ratio-neg}
\delta_t^{j,-} = \begin{cases}
0 & \delta_t^j  \geq 0\\
-\delta_t^j & \delta_t^j  < 0
\end{cases}
\end{equation}

It can be noticed that when a virtual arbitrage is conducted where some $e_t^{d,i}$ and $e_t^{c,i}$ are offset, it will activate the degradation damping with Equation \eqref{eq:degradation-damping} while there are no real physical processes causing degradation. This will be corrected in final profit calculation but in decision making process using optimizations we keep it as it is intended to restrict the virtual arbitrage.

Similar to Equation \eqref{eq:decision-f-revenue-end}, we reconstruct the diagonal matrices with the decomposed ratios from Equation \eqref{eq:ratio-pos} and \eqref{eq:ratio-neg}.
\begin{equation}
\Delta^+ = diag (
\delta_1^{J(1),+}, \dots , \delta_T^{J(1),+}, \dots, \delta_1^{J(|J|),+}, \dots, \delta_T^{J(|J|),+})
\end{equation}
\begin{equation}
\Delta^- = diag (
\delta_1^{J(1),-}, \dots , \delta_T^{J(1),-}, \dots, \delta_1^{J(|J|),-}, \dots, \delta_T^{J(|J|),-})
\end{equation}

The matrix of coefficient for degradation is the derived complying with the form of market modules:

\begin{equation*}
Cost^{degradation} = \begin{bmatrix}
Z^{I}~|~&Z^{I}~|~& \zeta (\Delta^{+} +\Delta^{-})
\end{bmatrix} \begin{bmatrix}
E^d \\ E^c \\ R
\end{bmatrix}
\end{equation*}

where,
\begin{equation*}
Z^I = \begin{bmatrix}
Z^{I(1)}~~|&\dots~~|&Z^i~~|&\dots~~|&Z^{I(|I|)}
\end{bmatrix}~~~
Z^i = \zeta \cdot I_{T\times T} ~~~ \forall i \in I
\end{equation*}

$I_{T \times T}$ is a ($T \times T$) identity matrix. 

\subsection{Technology simulation module}
\label{sec:tech-simulation-module}
The technology simulation is applied to determine the state of the system, which would be used primarily for calibration of technology constraints but also for \textit{ex-post} analysis.

\subsubsection{Energy Storage}
Regardless of the type of technology, an energy storage system consists of three functional units, i.e. power input, power output, and storage. Each function unit is associated with an efficiency, i.e. conversion efficiencies of charging, discharging and storage efficiency, denoted as $\eta_c$, $\eta_d$ and $\eta_s$ respectively.

Since the ramp up time for a typical storage system is neglectable comparing to the time resolution in our study, the state of power input and output are deemed as strictly following the operational plan without transient process.

For the state of storage, we define a term, $s$ (in MWh), which is the energy stored in the device, i.e. the State-of-Charge (SoC) multiplied by its maximum energy capacity. The state is determined using Equation \ref{eq:tech-ESS}.

\begin{equation}
\label{eq:tech-ESS}
s_t = \eta_s s_{t-1} + \eta_c (\sum_{i}^{i \in I} e_t^{c,i} + \sum_{j}^{j \in J}\delta_t^{j,-}r_t^j)- \frac{1}{\eta_d} (\sum_{i}^{i \in I} e_t^{d,i} + \sum_{j}^{j \in J}\delta_t^{j,+}r_t^j)
\end{equation} 

In order to formulate Equation \eqref{eq:tech-ESS} in matrix form, we first introduce a matrix denoted $H$:
\[
H
=
\begin{bmatrix}
\eta_s^0 & 0 & 0 &  \dots & 0 \\
\eta_s^1 & \eta_s^0 & 0 &  \dots & 0 \\
\eta_s^2 & \eta_s^1 & \eta_s^0 &  \dots & 0 \\
\vdots & \vdots & \vdots &  \ddots & \vdots \\
\eta_s^{T-1} & \eta_s^{T-2} & \eta_s^{T-3} & \dots & \eta_s^0 \\
\end{bmatrix}
\]
\newline
Then $M$ is used to construct $H^I$ and $H^J$ with a given pair of sets of market segments $I$ and $J$.
\begin{equation*}
H^I = \begin{bmatrix}
H^{I(1)}~~|&\dots~~|&H^i~~|&\dots~~|&H^{I(|I|)}
\end{bmatrix}~~~
H^i = H ~~~ \forall i \in I
\end{equation*}
\begin{equation*}
H^J = \begin{bmatrix}
H^{J(1)}~~|&\dots~~|&H^j~~|&\dots~~|&H^{J(|J|)}
\end{bmatrix}~~~
H^j = H ~~~ \forall j \in J
\end{equation*}
Finally, we can derive the matrix form of Equation \eqref{eq:tech-ESS}.
\begin{equation}
\label{eq:tech-ESS-M}
S = \eta_s H S_0 + \begin{bmatrix}
-\frac{1}{\eta_d} H^I~~|& \eta_c H^I~~|& H^J (-\frac{1}{\eta_d} \Delta^{+} + \eta_c \Delta^{-})
\end{bmatrix} X
\end{equation}
where, $S$ and $S_0$ are vectors for the temporal and initial state, respectively.
\begin{equation*}
S = \begin{bmatrix}
s_1~~s_2~~\dots~~s_T
\end{bmatrix}^T
\end{equation*}
\begin{equation*}
S_0 = \begin{bmatrix}
s_0~~s_0~~\dots~~s_0
\end{bmatrix}^T
\end{equation*}

In order to make it more compact, we reformulate Equation \eqref{eq:tech-ESS-M} as:

\begin{equation}
\label{eq:state-ESS-M-1}
S = \textit{\textbf{h}}_0 + \textit{\textbf{h}}~X
\end{equation}
where
\begin{equation}
\label{eq:state-ESS-M-2}
\textit{\textbf{h}}_0 =  \eta_s H S_0
\end{equation}
\begin{equation}
\label{eq:state-ESS-M-3}
\textit{\textbf{h}} = \begin{bmatrix}
-\frac{1}{\eta_d} H^I~~|& \eta_c H^I~~|& H^J (-\frac{1}{\eta_d} \Delta^{+} + \eta_c \Delta^{-})
\end{bmatrix}
\end{equation}
\subsubsection{Electric Vehicle}

Electric vehicle to grid systems are fundamentally battery energy storage systems in term of their physical dynamics. Therefore, they can be modeled generally using the same approach as in preceding paragraphs. However, there are several attributes that uniquely characterize electric vehicle to grid systems compared to normal battery storage:

\begin{itemize}
	\item The availability of an EV2G system, in terms of delivering both energy (in MWh) and capacity reserve (in MW), is dynamic rather than static, since the number of EVs connected in the power grid is changing all the time with the behaviors of plug-in/ plug-out.
	\item The energy stored in the system will be consumed not only for delivering our targeted services (arbitrage or balancing), but also for driving of EVs themselves. This part of costs will be implicitly captured by the revenue module using Equation \eqref{eq:module-revenue}, which will distort the real value of services provided for the grid. 
\end{itemize}

Therefore, two main modifications are made to adapt the model of ESSs for better representing the EV2G systems: 

\begin{enumerate}
	\item The EV2G system is modeled as a dynamic ESS by taking into consideration the connection/ disconnection of EVs to/ from the grids.
	\item The costs of energy consumed for driving are accounted and analyzed separately
\end{enumerate}

It shall be noticed with the dynamic storage model, only the overall state on the whole system level, i.e. the aggregation of all EVs in the system, is monitored and complied with the technological constraints. Performing simulation and optimization for each EV with a distributed approach is beyond the scope of this study. 

In order to transform the model for ESS to be dynamic in size and availability, we introduce additional terms to represent the number of EVs entering ($n_t^+$), leaving ($n_t^-$) and remain in ($n_t$) the system at each time step.
\begin{equation}
\label{eq:EV-number}
n_t = n_{t-1} + n_t^+ - n_t^-
\end{equation}

The energy stored in each EV while being plugged-in or plugged-out are denoted as $s_t^+$ and $s_t^-$, respectively. $n_t^+$, $n_t^-$, $s_t^+$ and $s_t^-$ can be determined statistically from real vehicle driving profiles. 

Thereby the state equation for an EV2G system is written as:
\begin{equation}
\label{eq:tech-EV}
\begin{aligned}
s_t = & \eta_s s_{t-1} + \eta_c (\sum_{i}^{i \in I} e_t^{c,i} + \sum_{j}^{j \in J}\delta_t^{j,-}r_t^j)- \frac{1}{\eta_d} (\sum_{i}^{i \in I} e_t^{d,i} + \sum_{j}^{j \in J}\delta_t^{j,+}r_t^j) \\
&+ s_t^+ n_t^+ - s_t^- n_t^-
\end{aligned}
\end{equation} 
\newline
The matrix form of Equation \eqref{eq:EV-number} is as following:
\begin{equation}
\label{eq:EV-number-M}
N = I_{T \times T} N_0 + L_{T \times T} N^+ -L_{T \times T} N^- 
\end{equation}
where, $L_{T \times T}$ is a ($T \times T$) identity lower triangular matrix. The rest matrices are defined as following
\begin{equation*}
N = \begin{bmatrix}
n_1~~n_2~~\dots~~n_T
\end{bmatrix}^T
\end{equation*}
\begin{equation*}
N_0 = \begin{bmatrix}
n_0~~n_0~~\dots~~n_0
\end{bmatrix}^T
\end{equation*}
\begin{equation*}
N^+ = \begin{bmatrix}
n_1^+~~n_2^+~~\dots~~n_T^+
\end{bmatrix}^T
\end{equation*}
\begin{equation*}
N^- = \begin{bmatrix}
n_1^-~~n_2^-~~\dots~~n_T^-
\end{bmatrix}^T
\end{equation*}
\begin{equation*}
S^+ = diag(s_1^+, s_2^+, \dots, s_T^+
)
\end{equation*}
\begin{equation*}
S^- = diag(s_1^-, s_2^-, \dots, s_T^-)
\end{equation*}
\newline
Analogously, translating Equation \eqref{eq:tech-EV} to matrix form leads to:

\begin{equation}
\begin{aligned}
S = &~ \eta_s H S_0 + H S^+ N^+ - H S^- N^- \\
&~+ \begin{bmatrix}
-\frac{1}{\eta_d} H^I~~|& \eta_c H^I~~|& H^J (-\frac{1}{\eta_d} \Delta^{+} + \eta_c \Delta^{-})
\end{bmatrix} X
\end{aligned}
\end{equation}
which can be reformulated as:
\begin{equation}
\label{eq:state-EV-M-1}
S = \textit{\textbf{h}}_0 + \textit{\textbf{h}}~X
\end{equation}
where
\begin{equation}
\label{eq:state-EV-M-2}
\textit{\textbf{h}}_0 =  \eta_s H S_0 + s^+ H N^+ - s^- H N^-
\end{equation}
\begin{equation}
\label{eq:state-EV-M-3}
\textit{\textbf{h}} = \begin{bmatrix}
-\frac{1}{\eta_d} H^I~~|& \eta_c H^I~~|& H^J (-\frac{1}{\eta_d} \Delta^{+} + \eta_c \Delta^{-})
\end{bmatrix}
\end{equation}

\subsection{Technology constraints}
\label{sec:tech-constraints}
The technology constraints are set to ensure the operation plan is fulfilled physically by the system.

\subsubsection{Energy storage}
Firstly, the charging/ discharging rate shall be bounded at its maximum rate ($\overline{r}$, assuming symmetric for charge and discharging).

\begin{equation}
0 \leq \frac{1}{\Delta t}\sum_{i}^{i \in I} e^{d,i}_t + \sum_{j}^{j \in J} r_t^j \leq \overline{r}~~~ \forall t \in T
\end{equation}

\begin{equation}
0 \leq \frac{1}{\Delta t}\sum_{i}^{i \in I} e^{c,i}_t + \sum_{j}^{j \in J} r_t^j \leq \overline{r}~~~ \forall t \in T
\end{equation}

It can be noticed that opposite movement of charging/ discharging in different markets are not offset in the constraints. This implies virtual arbitrageurs are not allowed to make deals that cannot be afforded physically although the physical systems are not actually activated.

Meanwhile, the energy stored is restricted as well.

\begin{equation}
0 \leq s_t \leq \overline{s}~~~ \forall t \in T
\end{equation}

Replacing $s_t$ using Equation \eqref{eq:tech-ESS}, the contraint is formulated as:
\begin{equation}
0 \leq \eta_s s_{t-1} + \eta_c (\sum_{i}^{i \in I} e_t^{c,i} + \sum_{j}^{j \in J}\delta_t^{j,-}r_t^j)- \frac{1}{\eta_d} (\sum_{i}^{i \in I} e_t^{d,i} + \sum_{j}^{j \in J}\delta_t^{j,+}r_t^j) \leq \overline{s}
\end{equation}

Applying the matrix format of the equations, we can get the constraints re-formulated the constraints of rates as:

\begin{equation}
- \frac{1}{\Delta t} \begin{bmatrix}
\overbrace{ I_{T\times T}|\dots|I_{T\times T}}^\text{|I|}|\overbrace{O_{T\times T}|\dots|O_{T\times T}}^\text{|I|}|\overbrace{ I_{T\times T}|\dots|I_{T\times T}}^\text{|J|}| 
\end{bmatrix} X \leq 0
\end{equation}

\begin{equation}
- \frac{1}{\Delta t} \begin{bmatrix}
\overbrace{ O_{T\times T}|\dots|O_{T\times T}}^\text{|I|}|\overbrace{I_{T\times T}|\dots|I_{T\times T}}^\text{|I|}|\overbrace{ I_{T\times T}|\dots|I_{T\times T}}^\text{|J|}| 
\end{bmatrix} X \leq 0
\end{equation}

\begin{equation}
\label{constraint:ESS-capacity}
\frac{1}{\Delta t} \begin{bmatrix}
\overbrace{ I_{T\times T}|\dots|I_{T\times T}}^\text{|I|}|\overbrace{O_{T\times T}|\dots|O_{T\times T}}^\text{|I|}|\overbrace{ I_{T\times T}|\dots|I_{T\times T}}^\text{|J|}| 
\end{bmatrix}X \leq \overline{R}
\end{equation}
\begin{equation}
\label{constraint:ESS-capacity-2}
\frac{1}{\Delta t} \begin{bmatrix}
\overbrace{ O_{T\times T}|\dots|O_{T\times T}}^\text{|I|}|\overbrace{I_{T\times T}|\dots|I_{T\times T}}^\text{|I|}|\overbrace{ I_{T\times T}|\dots|I_{T\times T}}^\text{|J|}| 
\end{bmatrix}X \leq \overline{R}
\end{equation}

where $O_{T \times T}$ is a $T \times T$ zero matrix and
\begin{equation*}
\overline{R} = \begin{bmatrix}
\overbrace{\overline{r}, \dots, \overline{r}}^\text{T}
\end{bmatrix}^T
\end{equation*}


The constraints of storage are formulated as:
\begin{equation}
\label{constraint:ESS-energy-1}
-\textit{\textbf{h}}~X \leq \textit{\textbf{h}}_0
\end{equation}
\begin{equation}
\label{constraint:ESS-energy-2}
\textit{\textbf{h}}~X \leq \overline{S} -  \textit{\textbf{h}}_0
\end{equation}
where, $\textit{\textbf{h}}$ and $\textit{\textbf{h}}_0$ are determined by Equation \eqref{eq:state-ESS-M-1} to \eqref{eq:state-ESS-M-3}, and
\begin{equation*}
\overline{S} = \begin{bmatrix}
\overbrace{\overline{s}, \dots, \overline{s}}^\text{T}
\end{bmatrix}^T
\end{equation*}

\subsubsection{Electric vehicle to grid}

The constraints for ESS are generally portable for the EV2G systems, by simplying re-using Equation \eqref{eq:state-EV-M-1} to \eqref{eq:state-EV-M-3} to derive $\textit{\textbf{h}}$ and $\textit{\textbf{h}}_0$, and replacing the upper bound limit in Equation \ref{constraint:ESS-capacity} with

\begin{equation}
\overline{R} = \overline{r}N
\end{equation}
where, $N$ is determined by Equation \eqref{eq:EV-number-M}. 




\section{Optimization Engine}
The performance of a flexibility resource depends primarily on the operation plan, which is represented as $X$ (Equation \ref{eq:decision-variable-1}). In order to value the market of technology vendors supplying flexibility to actors in power markets, we need to find reasonable operation patterns that simulate the behaviors of those players. For this sake, we employ an optimization engine. The value of market calculated with the results from optimization stands for the upper bound of market value.

The objective function of the optimization problem is formulated as:
\begin{equation}
\underset{X}{max} \left[ (1-\beta)\left(Revenue (X) - C^{degradation}(X)\right) - \beta CVaR(X) \right]
\end{equation}
where, X is the vector of decision variables (Equation \eqref{eq:decision-variable-1}), and $Revenue$, $C^{degradation}$ and $CVaR(X)$ are calculated using the equations in corresponding modules. $\beta$ is a weighting parameter with $\beta \in [0,1]$, which is used to study the trade-off between profit and risk.

The constraints have been introduced in the modules of market and technology constraints.

The optimization is implemented in MATLAB\textcopyright~and solved using Guobi optimizer. 

\section{Addtional measures for special cases}
\label{sec:special}

\subsection{Backcast technique to reduce the predictability of price}
As has been discussed in the literature review, many of the researches on arbitrage of flexibility in power markets assume the players have perfect foresights of future price movement, which would lead to an over-estimate of the real market value. Reducing the length of predictable window, using 'backcast' technique, and introducing stochastic programming are the usual choices to deal with this issue.

In this thesis, although the players would suffer risks of uncertain price movement with the introduction of stochastic part of price, they were still assigned with full foresight of the probability distribution. One may argue this is also unrealistic and could probably over-estimate the market potential. Therefore, by extending the work \cite{Drury2011} and \cite{Sioshansi2009}, we preformed a sensitivity analysis with reduced predictability using backcast.

We assume the way players predict the short-term forecast of future price is using the following equation:

\begin{equation}
\hat{\pi}_t = \hat{\pi}_{t-t_w} \cdot \frac{\sum_{t-t_w+1}^{t-t_d}\pi_{\tau}}{\sum_{t-t_w-t_w+1}^{t-t_w - t_d}\pi_{\tau}}
\end{equation}
where, $t_w$ is the time period of one week and $t_d$ is the time of one day. The future price is determined by taking the price curve shape of the day of last week and is adjusted by the 7-days average price level.

\subsection{Coupling day-ahead and real-time energy market}

When we value a case where player can participate in day-ahead and real-time (intraday, balancing) energy markets at the same time, an issue rises as they were assigned with full foresight and could easily leverage this advantage to make virtual arbitrage between day-ahead and real-time markets. Since the virtual arbitrage does not activate any physical process and purely benefited from the unrealistic foresight, it has to be constrained. Some researchers have also noticed this issue and used techniques such as put a proportional constraint of real-time volume to day-ahead volume \cite{Han2017} or deny reserved biddings between day-ahead and real-time market \cite{Berrada2016}.

In this thesis, the virtual arbitrage has already been damped by the degradation model as has been discussed in Section \ref{sec:cost} and restricted by the rate constraints in Section \ref{sec:tech-constraints}. Furthermore, we would perform a two-stage optimization where the day-ahead decisions will be made without knowing the real-time prices and the decisions for real-time market biddings will be determined afterwards to reflect the real market condition. We will compare the impact of virtual arbitrage in sensitivity analysis. 

\subsection{Dealing with non-energy-neutral signal for frequency control}
Providing frequency control is an attractive option for flexibility management as it is more profitable than energy arbitrage in current market context. However, a challenge of performing frequency control with non-generating flexibility sources is the non-energy-neutral signals of frequency regulation. If the control signal is not energy-neutral or not auto-corrected, it is not possible for a non-generating resource to provide service for an extended period due to the limited energy capacity. For example, a battery cannot absorb any more energy while it is fully charged and fail to continue delivering frequency control services.

Although some system operators have already implemented special energy neutral signals for the emerging flexibility resources, it is not a universal practice among the markets. 

In this study, we referred to the similar works \cite{Megel2017}\cite{Oudalov2007}\cite{Borsche2013}\cite{Jin2014} where the biased regulation signals are offset using external measure, e.g. via bilateral transactions or purchasing from the power markets. We assume that actors will purchase energy from the power market with real-time price to neutralize the regulation signal . 

\subsection{Final adjusted profit calculation}
As has been discussed above, we have introduced a list of treatments to better model the problem. However, some of the treatments would distort the perceived profits deviating from actual profits received by the actors, i.e. the differences exist between the value for decision making and for final accounting. Therefore, after performing the optimization, we would use the determined operation plan to re-calculate the profits to get the real values. 

~\newline

\textit{(Descriptions about Data has been moved to the chapter of case study as they are market-specific rather than generic.)}

\chapter{Case Studies}
\section{Analyzing the power market structures and business opportunities in select cases}

The superset of I is the set of selected energy market segments in different geographies:

\begin{equation*}
I \subseteq  \begin{cases}
\{Day~Ahead, Real~Time\} & PJM \\
\{Day~Ahead, Intraday, Balancing\} & Germany \\
\{Real~Time\} & NSW
\end{cases}
\end{equation*}

The superset of I is the set of selected reserve market segments in different geographies:

\begin{equation*}
J \subseteq  \begin{cases}
\{RegA, RegD, SR, NSR, DASR\} & PJM \\
\{PCR, SCR+, SCR-, TCR+, TCR-\} & Germany \\
\{Lower, Raise\} \times \{REG, 6SEC, 60SEC, 5MIN\} & NSW
\end{cases}
\end{equation*}

\subsection{PJM}
\subsubsection{Organization of PJM power markets}
Marketplaces
Timeline

\subsubsection{Players}
A Load Serving Entity (LSE), as is defined officially by PJM, is "any entity that has been granted authority or has an obligation pursuant to state or local law, regulation, or franchise to sell electric energy to end-users that are located within the PJM RTO. An LSE may be a Market Buyer or a Market Seller"\cite{PJM2017b}. Therefore, LSEs refer to all market participates in PJM who have rights and obiligation to act in all the power marketplaces of PJM, including the energy, capacity and ancillary services markets. 

Curtailment Service Providers (CSPs) are members in PJM markets specializing in demand response. A CSP is an intermitted agency that provides the end-user DR to the wholesale market. \cite{PJM2017b} \cite{Wang2015} The role of the CSP is actually a legacy product from the liberalization of retail markets in PJM. Once the retail competition began, PJM allowed LSEs to provide DR not only for their own customer but also for customers of other LSEs. The role of the CSP was created to facilitate the liberalization and competition. \cite{PJMInterconnection2017}

\subsubsection{Balancing mechanism}
submit offer - rebid - update information up to 65 mins - deviation charged with real-time

reviewed the participation, violating -> suspend activity, enter enforcement

LSE obiligate to purchase (or self-schedule) reserve, obiligation as a proportion to its contributing flow to the grid. \cite{PJM2017c} This incents liquidity in the market with competitions on both buyer's and seller's side. However, the obiligation does not reflect their actual needs.\cite{Wartsila2014}

CSP
intermitted agency
allowed to voluntarily respond to the LMP

\subsubsection{PJM DR}

PJM DR is the umbrella for all distributed energy resources, including DR, behind-the-meter generations, storage, etc. since PJM does not specify how the load is reduced. However, PJM DR program does not allow energy injection beyond the meter and receive wholesale compensation.\cite{PJMInterconnection2017}. This issue is currently under discussion in Special Market Implementation Committee meetings.

DR emergency fast changing over years \cite{Brown2015}
Since the DR in the wholesale market as a supply recouse will cause double payment issue where a customer may receive wholesale energy revenue and retail cost savings for the same MW of load reduction, PJM states that DR participation in the retail market on the demand side would be more ideal. And they are discussing to revisit the mechanism. Therefore, this value is not fully modeled in our study.

LSE
buyer or seller in Energy, and reserve market

\subsubsection{Identify business model}

\subsubsection{Accounting}





The real-time market price is applied for all deviations from day-ahead planned schedule, including Regulation, Primary and Supplementary Reserves.

\begin{equation*}
\pi_t^{e,j} = \pi_t^{e,i} ~~~~ i \in \{Real~Time\}, j \in \{RegD, RegA, SR, NSR, DASR\}
\end{equation*}

The capacity prices of reserves are computed using a complex algorithm, taking into account a list of specifications of the resrouce, e.g. the performance \& historical performance, benefits factor, milleage, etc. The detailed calculations can be found in appendix. As outputs, we will get deterministic values for $j \in \{RegA, SR, NSR, DASR\}$, and the upper and lower bounds, $\overline{\pi}_t^{r,j}$ and $\underline{\pi}_t^{r,j}$, for $i \in \{RegD\}$.

%Reg = RMCCP + RMPCP + LOC
%LOC = 0
%RMCCP = 
%RMPCP = Milleage
%Effective MW = BF * MW
%BF is determined with the average and upper, lower bounds

\subsection{Germany}
$\pi_t^{e,i}, i \in \{Balancing\}$, is the the price for balancing energy (reBAP), which exist only in Germany

$\pi_t^{r,j}$ and $\pi_t^{e,j}$ are based on principle of pay-as-bid. The weighted-average values are available in the datasets.

%Market participants in Germany are organised into balancing groups, known as Bilanzkreise (BK). BK can range from individual large generators to aggregations of smaller renewable generations, to a Stadwerke representing large portions of aggregated demand.

%Every balancing group operator is responsible for following a planned schedule with a 15-minute resolution. Deviations from the planned schedule are balanced physically by the TSOs and settled financially with the BK. There is a legal obligation on Bilanzkreise to balance their positions to the best of their ability.

Prices for balacning energy are unified across TSOs and determined according to the  balancing energy price settlement system (BK6-12-024) developed by Federal Network Agency (FNA) as of 01/12/2012.

\begin{equation}
reBAP = \frac{\sum net imbalance energy cost}{\sum net imbalance energy volume}
\end{equation}

\subsection{Australia-New South Walse}
The unit prices of reserve products, $\pi_t^{r,j}$ and $\pi_t^{e,j}$, are not available in datasets published by AEMO. Only weekly summary for total payment and recovery are provided. Due to the limits of available data, we are only able to perform calculations of total potential revenues, rather than thorough studies as in the other two geopraphies.

\section{Accounting rules and data preparation}


\section{Results and discussion}




\chapter{Conclusions and Outlook}
\label{conclusion}
%\input{conclusion}
%0.	Parametric analysis of battery capacity and PV power to estimate the optimal design of the building.
\section{Conclusions and implications}

In this thesis, based on a comprehensive study of  the academic research and the real-world market regimes, we developed an analytical framework for qualitative analysis and a techno-economic model for quantitative valuation of flexibility solutions. We further utilized them for case studies in three regimes, i.e. PJM in the US, Germany and NSW in Australia. 

Main conclusion and implications for technology vendors include:

\begin{enumerate}
	\item There are indeed no common rules among different market regimes but the methodology for analysis and valuation can be generalized:
	
	Using the framework and model developed in this thesis, the opportunities and challenges for flexibility solutions in different markets can be easily analyzed. This exercise shall be carried out for each specific case.
	
	\item Market regimes may have explicit and implicit impacts on value of flexibility solutions:
	
	Explicit impacts, mainly regarding the accessibility of certain technologies to power markets, can be generally identified through the qualitative analysis, while implicit impacts that are reflected on profitability might have to be studied via quantitative studies.
	
	\item Market potential for flexibility solution could be significantly attractive. However, right technology and business model are needed to unlock the value:
	
	Batteries are shown to be still too expensive and such situation is not likely to change in the near future even with rapid decreasing price. Solutions such as demand response without CAPEX could be more attractive in terms of profitability but aggregating distributed resources is more complex than operating centralized unit and may face more barriers due to market rules.
	
	\item The impacts of renewable penetration is verified but it still depends on the market regime:
	
	The overall generation mix and market rules such as negative pricing are the determinate factor for the impact of renewable penetration.
	
	
	%\item There are naturally more barriers due to legacy rules, in regimes with power pool arrangement than using  power exchange model, due to the bundled activities of on physical and market aspects. Therefore, technology vendors 
\end{enumerate} 

\section{Outlook and recommendation for future works}

Overall, we find that there are a significant abundance of literature that are related to this topic. However, few of them are indeed sharing the same perspective as us, that is to support business decision making for companies with international scope. Therefore, we believe this thesis where we spent enormous efforts to align different market regimes and establish a comprehensive view could be a good guidance for future researchers who are interested in this topic. 

Our work starts from a very broad scope and thus due to time limits, many of the aspects that were identified throughout the process cannot be fully incorporated into the study. This mean there is obviously space for further improvements and future works. Some recommendation for future works including:

\begin{enumerate}
	\item In Chapter \ref{ch:LitRev}, we sort out a comprehensive overview of major techniques used for quantitative valuation of flexibility solution. However, in order to keep agility and feasibility of the model, we adopt the relative simplified methods. Therefore, readers can refer to the analysis and test different methods for their effectiveness. Some examples including:
	\begin{itemize}
		\item Stochastic programming: as we have pointed out, there are three major stochastic terms related to flexibility solutions, i.e. price movement, frequency control signals and end-users' behaviors. Regarding the end-users' behaviors or more specifically the EV driving profiles, we believe a Markov chain model could be a good method by viewing the EV location profile as a stochastic process.
		\item Hybrid system: in this thesis, we did not simulated a mixed portfolio, while for aggregators such an exercise is necessary;
		\item Miscellaneous items like the price-maker effects, imperfect price foresight that have been fully elaborated in Chapter 2
	\end{itemize} 
	\item In this thesis, we implemented the market simulation module for forecasting long-term price trend in energy market. However, there are almost no literature found to make the similar exercise for frequency control prices. Although the frequency control markets are organized heterogeneously in different regimes as well as the pricing mechanisms, exercises for a single market shall still be valuable, since currently it seems to be black space.
	\item Improve the degradation cost modeling: we used a linear relationship between energy throughput and degradation while the modeling could be much more sophisticated discussed in Section \ref{sec:cost}.
	\item Implementation algorithm for evaluated performance of flexibility solution delivering frequency regulation: in some market as PJM, the actual performances are monitored and accounted for payments. However, this is not simulated in our study.
	%\item Finally, we believe there are few similar works that have an approach to study multiple markets and technologies in one framework, so we believe improvement on implementation of this model itself shall be a meaningful work. Examples of limitation and possible improvements include: determination of degradation cost, 
\end{enumerate}




\appendix
%\begin{landscape}
%\chapter{Model parameters}
%\label{sec:coefficients}
%\input{coefficients}
%\end{landscape}
\chapter{Terminology for frequency control service in different markets}
\label{app:terminology-frequency-control}

\chapter{Accounting rules and electricity market data preparation}
\label{sec:accounting-data-prepare}






\backmatter

% Enter your citations in a file thesis.bib and run BibTex.
%\nocite{koeppel} \nocite{writing_in_english} \nocite{goeschka}
%\nocite{oetiker} \nocite{kopka1}
%\newpage \addcontentsline{toc}{chapter}{Bibliography}
\bibliographystyle{unsrt}
\bibliography{thesis}

\end{document}
