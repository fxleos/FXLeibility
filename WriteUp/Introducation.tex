\chapter{Introduction}
\label{ch:introduction}
%\input{introduction}
\section{Background}

Background

Definition of flexibility

The challenges due to renewable penetration:

Traditional flexiblity from supply-side has limitations due to %Observations on market - increasing demands for flexibility due to renewable penetration

The increasing demand can be fulfilled in various means, including conventional methods like generation (gas turbine), tramsimission (grid extend), which normally requires vast investments on infrastructure. With the develop of technologies in ICT and batteries, new options are becoming increasingly feasbile %Observations on technology - reducing cost and increasing availability of technology

The push and pull from market demands and technology availability is leading the policy makers to review or even revise the regulatory framework which were established based on the  to allow non-discriminary participations of those new technologies. %Observation on regulation - regulatory changes, market redesign

Uncapping the potential

\section{Technologies: options for system flexibility provision}

\begin{itemize}
	\item supply-side flexibility
	\subitem Conventional power plant response
	\subitem Curtailment of variable renewable
	\item Energy Storage System (ESS)
	\subitem Battery Energy Storage System (BESS)
	\subitem Pumped Hydro Energy Storage (PHES)
	\subitem Compressed Air Energy Storage (CAES)
	\subitem Flywheel
	\item Demand Response (DR)
	\item Other
	\subitem Electric Vehicle to Grid (V2G)
	\subitem Electricity to Heat (E2H)
	\subitem Power to Gas (P2G) / Power to Hydrogen (P2H)
\end{itemize}

\section{Applications, benefits and business models}
\subsection{In liberalized market}

\subsubsection{Needs of different plyaers}

Player * Market * Application


\subsubsection{Energy Markets}


\subsubsection{Ancillary Service Markets}

\subsection{In vertically integrated market}


\section{Scope and research questions}

The target audience of this thesis is the management at Landis+Gyr on a high coporate level.

The ultimate goal is to provide references to support the audiences' strategic decision makings regarding flexibility management.

In order to achieve this, we conducted qualitative studies and developed quantitative models to identify: 1) the value of markets for flexiblity management

\begin{itemize}
	\item 
\end{itemize}

The goal of this thesis is to:

developed a robust modeling tool with moderate complexity so that it can not only provide results in current environment but can be also reused or easily revised to provide results in case of changes in the future.

based on the tool, make quantitative as well as quanlitative analysis to provide refer 

Purpose: providing references for strategic decision makings regarding flexibility management.

In order to make the analysis robust and reliable, we have built a techno-economic models which include the bottom-up dynamics of some key elements regarding the electricity markets and flexilibity technologies. 

However, it shall be noticed this thesis is not intended to serve for:

project developers to design a flexiblity system or make operating (including bidding) strategies of the system

policy makers to redesign the electricity market structure, rules or other policies

grid planners to understand the needs and options of flexibility in order to acheive system relability with lowest costs


Since the concept of flexiblity management is related to a great variety of technologies, applications and Landis+Gyr is positioning globally in various markets, the scope could be very broad. Nonetheless, in order to produce viable and reliable results with a solidily established techno-economic model, we have to make comprises. According to the relevance to Landis+Gyr's business, the scopes are defined as:

%\subsection{Scope of technologies}

%\subsection{Scope of applications}

%\subsection{Scope of benefits and business models}
The potential business model of Landis+Gyr is either to supply products to the customers to help them enable flexibility or to directly sell them flexible MWs as a service. In this case, we want to understand the value of each MW we enabled or sold. We assume Landis+Gyr will not directly partipate and trade in the power market, as it is going to place Landis+Gyr at the rival side of some customers in that market.

The value of flexibility will definitely vary according to the purpose, users' portfolio and operating strategies. 


%\subsection{Scope of markets}