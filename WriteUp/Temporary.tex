\section{Market-based modules}
\subsection{Revenue modules}

\subsection{Market simulation modules}
\begin{algorithm}
	\caption{Revenue modules}\label{RevenueModule}
	\KwIn{$\hat{\pi}^g, \kappa, \alpha, $}
	\KwOut{$\pi_t$}
	\KwResult{b}
	\For{$x \in X$}{
		s 
		
	}
\end{algorithm}




\section{Optimization}
Generally the objective function is formulated as maximizing the revenue which consists of revenues from selling both energy and reserve capacity net cost of purchasing energy.

\begin{equation}
\label{eq:obj-general}
\begin{aligned}
& \underset{e_t^d, e_t^c, r^{RU}_t, r^{RD}_t}{\text{maximize}}
& & \sum_t^{t \in T} \mathrm{Revenue_t} \\
& & &= \sum_t^{t \in T} [(p_t^{e, d}-C^d) e_t^d - (p_t^{e, c}+C^c) e_t^c ...\\
& & & ... + (p_t^{e, RU}-C^d) \delta_t^{RU} r_t^{RU} - (p_t^{e, RD}+C^c) \delta^{RD}_t r^{RD}_t ... \\
& & & ...  + p_t^{r, RU} r^{RU}_t + p_t^{r,RD} r^{RD}_t]\\
\end{aligned}
\end{equation}

$\delta^{RU}_t$ and $\delta^{RD}_t$ are usually calculated from a regional average of all resource:
\begin{equation*}
	\delta^{RU}_t = \frac{\sum_{i}^{i \in I} e_{t,i}^{RU}}{\sum_{i}^{i \in I} r_{t,i}^{RU}} = \frac{e_t^{RU, total}}{r_t^{RU, total}}
\end{equation*}

\begin{equation*}
\delta^{RD}_t = \frac{\sum_{i}^{i \in I} e_{t,i}^{RD}}{\sum_{i}^{i \in I} r_{t,i}^{RD}} = \frac{e_t^{RD, total}}{r_t^{RD, total}}
\end{equation*}

This revenue subjects to a list of constraints which can be categorized into two groups:
\begin{itemize}
	\item \textbf{Technology Constraints}: associated with the dynamics of the flexible asset, which are determined by their technical nature
	\item \textbf{Market Constrains}: associated with the market conditions including market rules and liquid volumes in the market
\end{itemize}

\section{Market constraints}

A constaint is taken to prevent the provision of flexible energy from replacing the base generation and charging of flexible energy from activating the peak loads. 
\begin{equation}
\label{eq:market-da-high}
e_t^{total} - e_t^{total, peak} \leq e_t^{d} - e_t^{c} + \delta_t^{RU} r_t^{RU} - \delta_t^{RD} r_t^{RD} \leq e_t^{total} - e_t^{total,base}
\end{equation}

For reserve market, the rule is 

\begin{equation}
\label{eq:market-ru-high}
r_t^{RU} \leq r_t^{RU,total}
\end{equation}

\begin{equation}
\label{eq:market-rd-high}
r_t^{RD} \leq r_t^{RD,total}
\end{equation}

\section{Different regimes}
\subsection{PJM}
\subsubsection{Day-ahead market}
Only $e_t^d$ and $e_t^c$ are to be optimized (terms for reserve are zero)

\begin{equation*}
p_t^{e, d} = p_t^{e, c} = p_t^{e,DA}
\end{equation*}

\subsubsection{Regulation market coupled with real-time market}

Since the total energy transaction associated with RegD and RegA are mixed in the datasets, an average ratio is taken:

\begin{equation*}
\delta^{Reg}_t = \delta^{RegD}_t =\delta^{RegA}_t   = \frac{e_t^{Reg, total}}{r_t^{Reg, total}}
\end{equation*}

We then seperate it into two components:

\begin{equation*}
\begin{cases} 
\delta^{Reg+}_t = \delta^{Reg}_t &\mbox{if } \delta^{Reg}_t \geq 0 \\ 
\delta^{Reg-}_t = -\delta^{Reg}_t & \mbox{if } \delta^{Reg}_t < 0 
\end{cases} 
\end{equation*}

\begin{equation*}
p_t^{e, d} = p_t^{e, c} = p_t^{e, RU}  = p_t^{e, RD}  = p_t^{e,RT}
\end{equation*}

Regulation market are divided to two segments, i.e. RegD and RegA. The price can be calculated as (assuming benefit factor being 1)
\begin{equation*}
p_t^{r, RegD} = (p_t^{RMCCP} + p_t^{RMPCP} * \gamma_t^{Mileage, RegD}) * \gamma_t^{Performance}
\end{equation*}

\begin{equation*}
p_t^{r, RegA} = (p_t^{RMCCP} + p_t^{RMPCP} * \gamma_t^{Mileage, RegA}) * \gamma_t^{Performance}
\end{equation*}

Lost opportunity credits are not accounted as our system is not a energy-providing resource.

Finally, \ref{eq:obj-general} is formulated as:
\begin{equation}
\begin{aligned}
& \underset{e_t^d, e_t^c, r^{RegD}_t, r^{RegA}_t}{\text{maximize}}
& & \sum_t^{t \in T} \mathrm{Revenue_t} \\
& & &= \sum_t^{t \in T} [(p_t^{e,RT}-C^d) (e_t^d + \delta_t^{Reg+} (r_t^{RegD} + r_t^{RegA})) ...\\
& & & ... -[(p_t^{e,RT}+C^c) (e_t^c + \delta_t^{Reg-} (r_t^{RegD} + r_t^{RegA})) ... \\
& & & ...  + p_t^{r, RegD} r^{RegD}_t + p_t^{r, RegA} r^{RegA}_t]\\
\end{aligned}
\end{equation}


\section{Technology Constraints}

\subsection{Electric Storage System}
\subsubsection{Energy contraints}

\begin{equation}
\label{eq:state-ess}
S_t = \eta_s S_{t-1} + \eta_c (e_t^c+\delta^{RD}_t r^{RD}_t)  - (1/\eta_d) (e_t^d+\delta^{RU}_t r^{RU}_t)
\end{equation}

The state of charge is limited so as to fulfill the need for deliver sufficient energy at time t:
\begin{equation}
\label{eq:limit-state}
0 \leq S_t \leq S_t^{max} %- \eta_c\delta_t^{RD} r^{RD}_t
\end{equation}

\subsubsection{Capacity constraints}

Capacity constraints guarantee there are enough margin in charging/ discharging capacity that has been committeed for reserve purpose:
\begin{equation}
\label{eq:limit-discharge-rate}
e_t^d / \Delta t + r^{RU}_t \leq r^{d, max}
\end{equation}

\begin{equation}
\label{eq:limit-charge-rate}
e_t^c / \Delta t + r^{RD}_t \leq r^{c, max}
\end{equation}

Non-negativity conditions:
\begin{equation}
e_t^d, e_t^c, r_t^{RU}, r_t^{RD} \geq 0
\end{equation}

All the constraints above shall be valid at any time:
\begin{equation*}
\forall t \in [1, 2, ..., T]
\end{equation*}